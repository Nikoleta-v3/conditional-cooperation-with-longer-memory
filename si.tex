\documentclass{article}

\usepackage{minitoc}
\usepackage{tabularx}
\usepackage{booktabs}
\usepackage{graphicx}
\usepackage{hyperref}
\usepackage{xcolor}
\usepackage{blkarray}
\usepackage{amsthm, amssymb, amsmath}
\usepackage{caption}
\usepackage{subcaption}
\usepackage{multirow}
\usepackage{multicol}
\usepackage[ruled,vlined]{algorithm2e}

\usepackage{natbib}
\bibliographystyle{abbrvnat}

\theoremstyle{definition}
\newtheorem{definition}{Definition}[section]
\newtheorem{theorem}{Theorem}[section]
\newtheorem{lemma}[theorem]{Lemma}
\newtheorem{conjecture}[theorem]{Conjecture}
\newtheorem{Prop}{Proposition}
\newtheorem{corollary}{Corollary}[theorem]

\usepackage[margin=2.5cm, includefoot, footskip=30pt]{geometry}
\pagestyle{plain}
\setlength{\parindent}{0em}
\setlength{\parskip}{1em}

\renewcommand{\baselinestretch}{1}

\usepackage{standalone}

\newtheorem{proposition}{Proposition}

\title{Reactive strategies with longer memory}

\author{Nikoleta E. Glynatsi, Ethan Akin, Martin Nowak, Christian Hilbe}
\date{}

\begin{document}

\maketitle

\section{Formal Model}

We consider infinitely repeated games among two players, player 1 and
player 2. Each round, they engage in the donation game with payoff matrix

\begin{equation} \label{Eq:DonationGame}
\left(
\begin{array}{cc}
b-c	&-c\\
b	&0
\end{array}
\right).
\end{equation}

Here $b$ and $c$ denote the benefit and the cost of cooperation, respectively. 
We assume $b\!>\!c\!>\!0$ throughout.
Therefore, payoff matrix~\eqref{Eq:DonationGame} is a special case of the
prisoner's dilemma with payoff matrix,

\begin{equation} \label{Eq:PrisonerDilemma}
    \left(
    \begin{array}{cc}
    R & S\\
    T & P
    \end{array}
    \right),
\end{equation}

where $T > R > S > P$ and $2 R > T + S$. Here, $R$ is the reward payoff of mutual
cooperation, $T$ is the temptation to defect payoff, $S$ is the sucker's payoff,
and $P$ is the punishment payoff for mutual defection.

We assume the repeated game is played infinitely many rounds, and that the
players' decisions only depend on the outcome of the previous $n$ rounds. To
this end, an {\it $n$-history for player $i \in \{1, 2\}$} is a string
$h^i=(a^i_{-n},\ldots,a^i_{-1})\!\in\!\{C,D\}^n$ where an entry $a^i_{-k}$
corresponds to player $i$'s action $k$ rounds ago. Let $H^i$ denote the space of
all $n$-histories for player $i$. Set $H^i$ contains $|H^i|=2^{n}$ elements. A
pair $h\!=\!(h^1,h^2)$ is called an {\it $n$-history of the game}. We use
$H=H^1\times H^2$ to denote the space of all such histories which contains
$|H|=2^{2n}$ elements.

A {\it memory-$n$} strategy is a vector $\mathbf{m}=(m_h)_{h\in
H}\in[0,1]^{2n}$. Each entry $m_h$ corresponds to the player's cooperation
probability in the next round, depending on the outcome of the previous $n$
rounds. One special case of memory$-n$ strategies are the round$-k-$repeat
strategies for some $1\!\le\!k\!\le\!n$. Player $1$ uses a {\it round-$k$-repeat
strategy} $\mathbf{m}^{k-\text{Rep}}$ if in any given round, the player chooses
the same action as $k$ rounds ago. That is, if the game's $n$-history is such
that,

$$
\begin{cases}
  m^{k-\text{Rep}}_h\!=\!1, \text{ if } a^1_{-k}\!=\!C\\[1em]
  m^{k-\text{Rep}}_h\!=\!0, \text{ if } a^1_{-k}\!=\!D.
\end{cases}
$$

Two additional special cases of memory$-n$ strategies that we will be discussing in
this work are, reactive$-n$ and self-reactive-$n$ strategies. A {\it
reactive$-n$ strategy} for player 1 is a vector
$\mathbf{p}=(p_h)_{h\in H^2} \in [0, 1]^{n}$. Each entry $p_h$ corresponds to
the player's cooperation probability in the next round, based on the co-player's
actions in the previous $n$ rounds. Therefore, reactive-$n$ strategies
exclusively rely on the co-player's $n$-history, independent of the focal
player's own actions. On the other hand, {\it self-reactive-$n$}
strategies only consider the focal player's own $n$-history, and ignore the
co-player's. Formally, a self-reactive-$n$ strategy for player 1 is a vector
$\mathbf{\tilde{p}} = (\tilde{p}_h)_{h \in H^1} \in [0, 1] ^ {n}$. Each entry
$\tilde{p}_h$ corresponds to the player's cooperation probability in the next,
depending on the player's own actions in the previous $n$ rounds.
From hereon, we will use the notations $\mathbf{m}, \mathbf{p}, \text{ and }
\mathbf{\tilde{p}}$ to denote memory-$n$, reactive-$n$, and self-reactive-$n$
strategies.

Let players 1 and 2 use memory$-n$ strategies $\mathbf{m^{1}}$ and
$\mathbf{m^{2}}$. Then one can represent the interaction as a Markov chain with
possible states $h \in H$. Assume that the present round is given by \(h =
(h^{1}, h^{2})\), the probability that one round later \(\tilde{h}\) is observed
is given by the product,

$$
  M_{h, \tilde{h}} = \prod_{i=1}^{2} x^{i}
$$

where,

$$
x^i = 
\begin{cases}
  m^{i}_{h} & \text{ if } \tilde{\alpha}^i_{-1} = C \text{ and } \tilde{\alpha}^i_{-t} = \alpha^i_{-t + 1} \text{ for all other } \tilde{\alpha}^i_{-t}\\
  1 - m^{i}_{h} & \text{ if } \tilde{\alpha}^i_{-1} = D \text{ and } \tilde{\alpha}^i_{-t} = \alpha^i_{-t + 1} \text{ for all other } \tilde{\alpha}^i_{-t}\\
  0 & \text{ if } \tilde{\alpha}^i_{-t} \neq  \alpha^i_{-t + 1} \text{ for some } 2 \leq t \leq n.
\end{cases}
$$

Let $\mathbf{v}=(v_h)_{h\in H}$ be an invariant distribution of this Markov
chain for which $\mathbf{v} = \mathbf{v} \cdot M$.

% Previously, the work of~\citep{akin:EGADS:2016} characterized all partner strategies in the
% case of memory-one strategies. For higher memory values ($n > 1$), a few works
% (\citep{hilbe:PNAS:2017}) have managed to characterize subsets of memory$-n$ partner
% strategies. This difficulty arises from the fact that as memory increases,
% obtaining analytical results becomes more challenging. In this work, we focus on
% reactive strategies instead of memory$-n$ strategies. Reactive strategies, a
% subset of memory$-n$ strategies, are formally introduced in Section~\ref{section:reactive_strategies}. We
% characterize all reactive partner strategies for $n = 2$ and $n = 3$, and present a
% series of results starting from Section~\ref{section:self_reactive_sufficiency}. In the following section, we will
% discuss a series of results for the case of memory$-n$.

\subsection{An Extension of Akin's Lemma}

The work of~\citep{akin:EGADS:2016} focuses on memory-one strategies. In the
case of $n=1$, a memory-one strategy is represented by the vector
\(\mathbf{m}=(m_{CC}, m_{CD}, m_{DC}, m_{DD})\). Let player $1$ use the
memory-one strategy $\mathbf{m}$ when interacting with player $2$. This leads to
a sequence of distributions \(\{\mathbf{v}^{t}, t = 1, 2, ...\}\) with
\(\mathbf{v}^{t}\) representing the distribution over the states in the
\(t^{\text{th}}\) round of the game. Let $\mathbf{v} = (v_{CC}, v_{CD}, v_{DC}, v_{DD})$ be an associated
stationary distribution of the interaction. In his work Akin shows that,

\begin{align}
  \lim_{n \rightarrow \infty} \frac{1}{n} \sum_{t=1}^{n} \mathbf{v}^{t} \cdot (\mathbf{m} - \mathbf{m}^{1-\text{Rep}}) & = 0, \text{ and therefore } \mathbf{v} \cdot (\mathbf{m} - \mathbf{m}^{1-\text{Rep}}) = 0.
\end{align}

With the same method as in \citep{akin:EGADS:2016}, one can derive a generalized
version of his result. Namely the generalized version is given by
Lemma~\ref{lemma:AkinGeneralised}.

\begin{lemma}[Generalized Akin Lemma]\label{lemma:AkinGeneralised}
Let player $1$ use a memory-$n$ strategy, and let player $2$ use any arbitrary
strategy. The interaction between the two players leads to a sequence of
distributions \(\{\mathbf{v}^{t}, k = 1, 2, ...\}\) with \(\mathbf{v}^{t}\)
representing the distribution over the states in the \(t^{\text{th}}\) round of
the game. Let $\mathbf{v}$ be an associated stationary distribution of the
interaction.
Then for each $k$ with $1\!\le\!k\!\le\!n$, the invariant distribution $\mathbf{v}$
satisfies the following relationship,

\begin{equation} \label{Eq:AkinsLemma}
\mathbf{v} \cdot (\mathbf{m}-\mathbf{m}^{k-\text{Rep}}) \!=\! \sum_{h\in H} v_h (m_h-m_h^{k-\text{Rep}}) = 0.
\end{equation}

\end{lemma}

\begin{proof}
Let the probability that player $1$ cooperated in the \(n^{\text{th}}\) round be
denoted as \(v_{\text{C}}^{n}\). Let \(v_{\text{C}}^{n}\) be defined as the
probability that player $1$ played \(C\), \(k \, (1\leq k\leq n)\) rounds ago. Then,

$$
v_{\text{C}}^{n} = \sum_{h \in H} y_{h}, \quad \text{ where } \quad y_h = 
\begin{cases}
 u_{h} & \text{ if } \alpha^1_{-k} = C \\
     0 & \text{ if } \alpha^1_{-k} = D.
\end{cases}
$$

Equivalently,

$$
v_{\text{C}}^{n} = \mathbf{v}^{n} \cdot \mathbf{m}^{k - \text{Rep}}.
$$

The probability that player $1$ cooperates in the \((n + 1)^{th}\) round,
denoted by \(v_{\text{C}}^{n + 1}\) = \(\mathbf{v}^{n} \cdot \mathbf{m}\). Hence,

\begin{equation*}
  v_{\text{C}}^{n + 1} - v_{\text{C}}^{n} = \mathbf{v}^{n} \cdot \mathbf{m} - \mathbf{v}^n \cdot \mathbf{m}^{k - \text{Rep}}
  =  \mathbf{v}^{n} \cdot (\mathbf{m} - \mathbf{m}^{k - \text{Rep}}).
\end{equation*}

This implies,

\begin{equation}
\sum^{n}_{t=1} \mathbf{v}^{t} \cdot (\mathbf{m} - \mathbf{m}^{k - \text{Rep}}) = \sum^{n}_{t=1} v_{\text{C}}^{t + 1} - v_{\text{C}}^{t} \quad \Rightarrow \quad \sum^{n}_{t=1} \mathbf{v}^{t} \cdot (\mathbf{m} - \mathbf{m}^{k - \text{Rep}}) =  v_{\text{C}}^{n + 1} - v_{\text{C}}^{1}.
\end{equation}

As the right side has absolute value at most 1,

\begin{equation}
\lim_{n \rightarrow \infty} \frac{1}{n} \sum^{n}_{t=1} \mathbf{v}^{t} \cdot (\mathbf{m} - \mathbf{m}^{k - \text{Rep}}) = 0.
\end{equation}
\end{proof}

The intuition for this result is that $\mathbf{v}\cdot \mathbf{m}$ and all
$\mathbf{v}\cdot \mathbf{m}^{k-\text{Rep}}$ are just different, but equivalent,
expressions for player $1$'s average cooperation rate. More specifically,
\(\mathbf{m}^{1-\text{Rep}} = \mathbf{m}^{2-\text{Rep}} = \dots =
\mathbf{m}^{n-\text{Rep}}\) correspond to the intuition that it does not matter
which of the past $n$ rounds we use to define cooperation rate.


\subsection{Payoffs and Further Definitions}

The invariant distribution $\mathbf{v}$ can also be used to define the long term
payoffs of the players'. Let $\mathbf{S}^k = (S_h^k)_{h\in H}$ denote the vector
that returns for each $h$ the one-shot payoff that player $1$ obtained $k$
rounds ago,

\begin{equation}
    S_h^k = \left\{
    \begin{array}{cl}
    b-c	&\text{if}~ a_{-k}^1=C~\text{and}~ a_{-k}^2=C\\
    -c	&\text{if}~ a_{-k}^1=C~\text{and}~ a_{-k}^2=D\\
    b	&\text{if}~ a_{-k}^1=D~\text{and}~ a_{-k}^2=C\\
    0	&\text{if}~ a_{-k}^1=D~\text{and}~ a_{-k}^2=D
    \end{array}
    \right.
\end{equation}

Then we can define player $1$'s repeated-game payoff $s_{\mathbf{m^{1}},\mathbf{m^{2}}}$ as

\begin{equation} \label{Eq:Payoff}
s_{\mathbf{m^{1}},\mathbf{m^{2}}}  = \mathbf{v}\cdot \mathbf{S}^1 = \mathbf{v}\cdot \mathbf{S}^2 = \ldots = \mathbf{v} \cdot \mathbf{S}^n.
\end{equation}

The equalities $\mathbf{v}\cdot \mathbf{S}^1 = \ldots = \mathbf{v} \cdot
\mathbf{S}^n$ correspond to the intuition that it does not matter which of the
past $n$ rounds we use to define average payoffs. This is an immediate result of
Lemma~\ref{lemma:AkinGeneralised}. The payoffs of the players depend on both
players' cooperations, and since their cooperation can be defined as having
occurred in any of the last $n$ turns, the payoffs can also be expressed
analogously. The payoff $s_{\mathbf{m^{2}},\mathbf{m^{1}}}$ of player $2$ can be
defined in a similar manner.

Let's provide definitions for some additional terms that will be used in this
manuscript.

\begin{definition}[Nash Strategies]
A strategy $\mathbf{m}$ for player $1$, is a \textit{Nash
strategy} if,

\begin{equation}\label{Eq:Nash}
    s_{\mathbf{m'},\mathbf{m}} \leq s_{\mathbf{m},\mathbf{m}} \;\forall \; \mathbf{m'}.
\end{equation}
\end{definition}

\begin{definition}[Nice Strategies] A player's strategy is \textit{nice}, if
the player is never the first to defect. A nice strategy against itself receives
the mutual cooperation payoff, $(b - c)$.
\end{definition}

\begin{definition}[Partner Strategies]
A \textit{partner strategy} is a strategy which is both nice and Nash.
\end{definition}

Partners strategies are of interest because they are strategies that strive to
achieve the mutual cooperation payoff of $(b - c)$ with their co-player.
However, if the co-player doesn't cooperate, they are prepared to penalize them
with lower payoffs. Partner strategies, by definition, are best responses to
themselves~\citep{Hilbe:GEB:2015}. All partner
strategies are Nash strategies, but not all Nash strategies are partner
strategies.

To check whether a strategy \(\mathbf{m}\) is Nash, one has to check
Eq.~\eqref{Eq:Nash} against all possible memory-$n$ strategies. However the
works of~\citep{mcavoy:PRSA:2019, Mcavoy:PNAS:2022} have shown that for $n=1$ it
is sufficient to check against all pure memory-one strategies. Here we
generalize this result to any $n$.

\begin{lemma}[Monotinic Payoffs]\label{lemma:monotonic_payoffs}
$\dots$
\end{lemma}

\begin{lemma}[Sufficiency of Pure Strategies]\label{lemma:sufficiency_pure_strategies}
$\dots$
\end{lemma}

\section{Tit For Tat and Generous Tit For Tat}

Based on Lemma~\ref{lemma:AkinGeneralised}, we can derive a theory of
zero-determinant strategies analogous to the case of memory-one strategies. In
the following, we say a memory-$n$ strategy $\mathbf{m^1}$ is a zero-determinant
strategy if there are $k_1$, $k_2$, $k_3$ and $\alpha$, $\beta$, $\gamma$ such
that $\mathbf{m^1}$ can be written as

\begin{equation} \label{Eq:DefZD}
\mathbf{m^1} = \alpha \mathbf{S}^{k_1} + \beta \mathbf{\tilde{S}}^{k_2} + \gamma \mathbf{1} + \mathbf{m}^{k_3-\text{Rep}},  
\end{equation} 
where $\mathbf{1}$ is the vector for which every entry is 1. By Akin's Lemma and the definition of payoffs,
\begin{equation} \label{Eq:PayoffZD}
0 = \mathbf{v} \cdot  (\mathbf{m^1} - \mathbf{m}^{k_3-\text{Rep}}) = \mathbf{v} \cdot (\alpha \mathbf{S}^{k_1} + \beta \mathbf{\tilde{S}}^{k_2} + \gamma \mathbf{1} ) = \alpha s_{\mathbf{m^1}, \mathbf{m^2}} + \beta s_{\mathbf{m^2}, \mathbf{m^1}} + \gamma. 
\end{equation}

That is, payoffs satisfy a linear relationship. 

One interesting special case arises if $k_1\!=\!k_2\!=\!k_3\!=:\!k$ and $\alpha
= -\beta =1/(b\!+\!c)$ and $\gamma=0$. In that case, the formula
\eqref{Eq:DefZD} yields the strategy

\begin{equation}
m_h = \left\{
\begin{array}{ll}
1	&\text{if}~~a^2_{-k}=C\\
0	&\text{if}~~a^2_{-k}=D
\end{array}
\right.
\end{equation}

That is, this strategy implements Tit-for-Tat (for $k\!=\!1$) or delayed
versions thereof (for $k\!>\!1$). These strategies are partners strategies that
also satisfy a stronger relationship. By Eq.~\eqref{Eq:PayoffZD}, the enforced
payoff relationship is $s_{\mathbf{m^1}, \mathbf{m^2}}\!=\! s_{\mathbf{m^2},
\mathbf{m^1}}$.

Another interesting special case arises if  $k_1\!=\!k_2\!=\!k_3\!=:\!k$ and
$\alpha\!=\!0$, $\beta\!=\!-1/b$, $\gamma\!=\!1\!-\!c/b$. In that case
Eq.~\eqref{Eq:DefZD} yields the strategy

\begin{equation}
m_h = \left\{
\begin{array}{ll}
1	&\text{if}~~a^2_{-k}=C\\
1-c/b	&\text{if}~~a^2_{-k}=D
\end{array}
\right.
\end{equation}

That is, the generated strategy is GTFT (if $k\!=\!1$), or delayed versions
thereof (for $k\!>\!1$). By Eq.~\eqref{Eq:PayoffZD}, the enforced payoff
relationship is $s_{\mathbf{m^2}, \mathbf{m^1}}\!=\!b\!-\!c$. In particular, these
strategies are partner strategies.\\

\section{Reactive Partner Strategies}\label{section:reactive_strategies}

We assume in the following, that $1$ adopts reactive$-n$ strategy $\mathbf{p}$,
and player $2$ adopts an arbitrary memory-$n$ strategy $\mathbf{m}$. We define
the following marginal distributions with respect to the possible $n$-histories
of player~$2$,

\begin{equation}\label{Eq:marginal_distributions}
\displaystyle v^2_{h} = \sum_{h^1\in H^1} v_{(h^1, h)}.
\end{equation}

These entries describe how often we observe player $2$ to choose actions $h^2$,
in $n$ consecutive rounds (irrespective of the actions of player $1$). Note
that,

\begin{equation}\label{eq:normalization_marginal_distributions}
  \displaystyle \sum_{h \in H^2} v^2_{h} = 1.
\end{equation}

Let $\mathbf{p^{k - \text{Rep}}}$ be a reactive round-$k$-repeat strategy. Then
the cooperation rate of player $2$, denoted as $\rho_\mathbf{m}$, is given by,

\begin{equation}\label{Eq:coplayer_cooperation_expr}
  \rho_\mathbf{m} = \sum_{h \in H^{2}} v^q_{h} \cdot p^{1 - \text{Rep}}_{h} \; = \; \sum_{h \in H^{2}} v^q_{h} \cdot p^{2 - \text{Rep}}_{h} \; = \; \dots \; = \sum_{h \in H^{2}} v^q_{h} \cdot p^{n - \text{Rep}}_{h}.
\end{equation}

The cooperating rate of player $1$ can also be expressed in terms of $v^2_{h}$ as,

\begin{equation} \label{Eq:rhop_alln}
  \begin{array}{lll}
    \rho_\mathbf{p} &= &\displaystyle \sum_{h \in H^2} v^2_{h} \cdot p_{h}.
  \end{array}
\end{equation}

Because we consider simple donation games, we note that these two quantities,
$\rho_\mathbf{m}  ~~and~~ \rho_\mathbf{p},$ are
sufficient to define the payoffs of the two players,

\begin{equation} \label{Eq:payoff}
  \begin{array}{lll}
  s_{\mathbf{p}, \mathbf{m}}  =  b\, \rho_\mathbf{m} - c\, \rho_\mathbf{p}\\
  s_{\mathbf{m}, \mathbf{q}} = b\, \rho_\mathbf{p} - c\, \rho_\mathbf{m}.
  \end{array}
\end{equation}

\subsection{Sufficiency of Self reactive strategies}\label{section:self_reactive_sufficiency}

\cite{press:PNAS:2012} discussed the case where one player uses a memory-one
strategy and the other player employs a longer memory strategy. They
demonstrated that the payoff of the player with the longer memory is exactly the
same as if the player had employed a specific shorter-memory strategy,
disregarding any history beyond what is shared with the short-memory player.
Here we show a result that follows a similar intuition: if there is a part of
history that one player does not observe, then the co-player gains nothing by
considering the history not shared with the reactive player.

\begin{lemma}\label{lemma:self_reactive_sufficiency}
  Let $\mathbf{p}$ be a reactive$-n$ strategy for player $1$. Then, for any
  memory$-n$ strategy $\mathbf{m}$ used by player $2$, player $1$'s score is
  exactly the same as if $2$ had played a specific self-reactive memory-$n$
  strategy $\mathbf{\tilde{p}}$.
\end{lemma}

\begin{proof}
$\dots$
\end{proof}

Lemma~\ref{lemma:sufficiency_pure_strategies} states that for a memory-$n$
strategy to be Nash, then condition~\eqref{Eq:Nash} has to hold against all pure
memory-$n$ strategies. This also applies to the case where the player adopts a
reactive-$n$ strategy. However, for the case of a reactive player here we show a
stronger result. More specifically, combining
Lemma~\ref{lemma:sufficiency_pure_strategies} and
Lemma~\ref{lemma:self_reactive_sufficiency} leads to the focal result,

\begin{lemma}\label{lemma:nash_against_pure_self_reactive}
A reactive-$n$ strategy $\mathbf{p}$ for player $1$, is a \textit{Nash strategy} if,

\begin{equation}\label{Eq:NashReactive}
    s_{\mathbf{\tilde{p}}, \mathbf{p}} \leq s_{\mathbf{p},\mathbf{p}} \;\forall \; \tilde{p} \in \tilde{P}.
\end{equation}

where,

$$\tilde{P} = \{(\tilde{p}_h): \tilde{p}_h \in \{0, 1\} \; \forall \; h \in H^2\}.$$

Thus, $\tilde{P}$ is the set of all pure self-reactive-$n$ strategies.
\end{lemma}

Lemma~\ref{lemma:nash_against_pure_self_reactive} implies that the space
of strategies we need to check against is even more constrained in the
case of reactive strategies. This has a huge implication on the computational
complexity of finding Nash strategies. In particular, the number of strategies
one has to check against is reduced from $2^{2^{2n}}$ to $2^{2n}$.

Another result of Lemma~\ref{lemma:self_reactive_sufficiency} is that we can
retrieve the marginal distributions of the co-player's actions
(Eq~\eqref{Eq:marginal_distributions}) without having to consider the transition
matrix $M$. For now, one has to calculate the transition matrix \(M\) for two
given players and calculate the stationary distribution of this $2^{2n} \times
2^{2n}$ matrix. However, since the co-player is playing a self-reactive strategy
then the co-player's action only rely on his/her actions, and thus one can model
this as a Markov process with states $h^2 \in H^2$ and a transition matrix
\(\tilde{M}\). Let \(h^2=((a^2_{-n},\ldots,a^2_{-1}))\) be the state in the
current round. The probability that in the next turn
\(\tilde{h}^2=((\tilde{a}^2_{-n},\ldots,\tilde{a}^2_{-1}))\) is observed is
given by,

$$
\tilde{M}_{h^2, \tilde{h}^2} = 
\begin{cases}
  \tilde{p}_{h^2} & \text{ if } \tilde{\alpha}^2_{-1} = C \text{ and } \tilde{\alpha}^2_{-t} = \alpha^2_{-t + 1} \text{ for all other } \tilde{\alpha}^2_{-t}\\
  1 - \tilde{p}_{h^2} & \text{ if } \tilde{\alpha}^2_{-1} = D \text{ and } \tilde{\alpha}^2_{-t} = \alpha^2_{-t + 1} \text{ for all other } \tilde{\alpha}^2_{-t}\\
  0 & \text{ if } \tilde{\alpha}^2_{-t} \neq  \alpha^2_{-t + 1} \text{ for some } 2 \leq t \leq n.
\end{cases}
$$

The stationary distribution of this Markov chain denoted as $\mathbf{\tilde{v}}$
has the property that,

$$
\tilde{u}_h = u^2_{h}.
$$

This results that the payoffs of the players can now be calculated more
efficiently.



\subsection{Reactive-Two Partner Strategies}\label{section:reactive_two_partner_strategies}

In this section, we focus on the case of $n=2$. Reactive-two strategies are denoted as a vector
$\mathbf{p}=(p_{CC}, p_{CD}, p_{DC}, p_{DD})$ where $p_{CC}$ is the
probability of cooperating in this turn when the co-player cooperated in the
last 2 turns, $p_{CD}$ is the probability of cooperating given that the
co-player cooperated in the second to last turn and defected in the last, and so
forth. A nice reactive-two strategy is represented by the vector $\mathbf{p}=(1,
p_{CD}, p_{DC}, p_{DD})$.

\begin{theorem}[``Reactive-Two Partner Strategies'']\label{theorem:reactive_two_partner_strategies}
A nice reactive-two strategy $\mathbf{p}$, is a partner strategy if and only if,
the strategy entries satisfy the conditions:

\begin{equation}\label{eq:two_bit_conditions}
  \displaystyle p_{DD} < 1\!-\! \frac{c}{b}  ~~and~~ \displaystyle \frac{p_{CD} + p_{DC}}{2} < 1- \frac{1}{2} \cdot \frac{c}{b}.
\end{equation}
\end{theorem}

There are two independent proffs of
Theorem~\ref{theorem:reactive_two_partner_strategies}. The first prove is
in line with the work of~\citep{akin:EGADS:2016}, and the second one relies on
the sufficiency of self-reactive strategies. We discuss both proofs
in the Appendix~\ref{appendix:proofs_reactive_two_partner_strategies}.

\subsection{Reactive-Three Partner Strategies}\label{section:reactive_three_partner_strategies}

In this section, we focus on the case of $n=3$. Reactive-three strategies are
denoted as a vector 

$$\mathbf{p}=(p_{CCC}, p_{CCD}, p_{CDC}, p_{CDD}, p_{DCC}, p_{DCD}, p_{DDC}, p_{DDD})$$

where $p_{CCC}$ is the probability of cooperating in round $t$ when the
co-player cooperates in the last 3 rounds, $p_{CCD}$ is the probability of
cooperating given that the co-player cooperated in the third and second to last
rounds and defected in the last, and so forth. A nice reactive-three strategy
has $p_{CCC} = 1$.

\begin{theorem}[``Reactive-Three Partner Strategies'']\label{theorem:reactive_three_partner_strategies}
A nice reactive-three strategy $\mathbf{p}$, is a partner strategy if and only if,
the strategy entries satisfy the conditions:

\begin{align}\label{eq:three_bit_conditions}
  \begin{split}
  \frac{p_{CCD} + p_{CDC} + p_{DCC}}{3} & < 1\!-\! \frac{1}{3} \cdot \frac{c}{b} \\
  \frac{p_{CDD} + p_{DCD} + p_{DDC}}{3} & < 1\!-\! \frac{2}{3} \cdot \frac{c}{b} \\
  p_{DDD} & < 1\!-\! \frac{c}{b} \\
  \frac{p_{CCD} + p_{CDD} + p_{DCC} + p_{DDC}}{4}  & < 1\!-\! \frac{1}{2} \cdot \frac{c}{b}  \\
  \frac{p_{CDC} + p_{DCD}}{2} & < 1\!-\! \frac{1}{2} \cdot \frac{c}{b}
  \end{split}
\end{align}
\end{theorem}

Once again, there are two independent proves of
Theorem~\ref{theorem:reactive_three_partner_strategies}, and we 
discuss both proofs in the Appendix~\ref{appendix:proofs_reactive_three_partner_strategies}.

\subsection{Reactive Counting Strategies}

A special case of reactive strategies is reactive counting strategies. These are
strategies that respond to the co-player's actions, but they do not distinguish
between when cooperations/defections occurred; they solely consider the count of
cooperations in the last $n$ turns. A reactive-$n$ counting strategy is represented
by a vector $\mathbf{r}=(r_i)_{i \in \{n, n -1, \dots, 0\}}$, where the entry \(r_i\)
indicates the probability of cooperating given that the co-player cooperated
\(i\) times in the last \(n\) turns.

Reactive-two counting strategies are denoted by the vector $\mathbf{r}=(r_2,
r_1, r_0)$. We can characterise partner strategies among the reactive-two
counting strategies by setting $r_2 = 1$, and $p_{CD} = p_{DC} = r_1$ and
$p_{DD} = r_0$ in conditions~\eqref{eq:two_bit_conditions}. This gives us the
following result.

\begin{corollary}
A nice reactive-two counting strategy $\mathbf{r} = (1, r_1, r_0)$ is a partner strategy if and only if,

\begin{equation}\label{eq:counting_two_bit_conditions}
  \displaystyle r_1 < 1-\frac{1}{2} \cdot \frac{c}{b} ~~and~~ r_0 < 1\!-\! \frac{c}{b}.
\end{equation}
\end{corollary}

Reactive-three counting strategies are denoted by the vector $\mathbf{r}=(r_3,
r_2, r_1, r_0)$. We can characterise partner strategies among reactive-three
counting strategies by setting $r_3 = 1$, and $p_{CCD} = p_{CDC} = p_{DCC} =
r_2, p_{DCD} = p_{DDC} = p_{CDD} = r_1$ and $p_{DDD} = r_0$ in
conditions~\eqref{eq:three_bit_conditions}. This gives us the following result.

\begin{corollary}
A nice reactive-three counting strategy $\mathbf{r} = (1, r_2, r_1, r_0)$ is a partner strategy if and only if,

\begin{equation}\label{eq:counting_three_bit_conditions}
  \displaystyle r_2 < 1- \frac{1}{3} \cdot \frac{c}{b}, \quad r_1 < 1- \frac{2}{3} \cdot \frac{c}{b} ~~and~~ r_0 < 1\!-\! \frac{c}{b}.
\end{equation}
\end{corollary}

In the case of counting reactive strategies, we generalize to the case of $n$.

\begin{corollary}[``Reactive-Counting Partner Strategies'']\label{corollary:reactive_counting_partner_strategies}
A nice reactive-$n$ counting strategy $\mathbf{r}=(r_i)_{i \in \{n, n-1, \dots, 0\}}$,
is a partner strategy if and only if:

\begin{equation}
  r_{n - k} < 1 - \frac{k}{n} \cdot \frac{c}{b}, \text{ for } k \in \{1, 2, \dots, n\}.
\end{equation}

\end{corollary}

\section{Prisoner's Dilemma}

So far we have focused on a special case of the Prisoner's Dilemma, the donation
game. In this section we show that the results of Sections~\ref{section:reactive_two_partner_strategies}
and~\ref{section:reactive_three_partner_strategies} can be generalized
for the iterated Prisoner's Dilemma.

We assume that player $1$ uses a reactive-$n$ strategy \(\mathbf{p}\), and that
the co-player uses a self-reactive strategy \(\mathbf{\tilde{p}}\). The interaction
can be model as a Markov proceed with a stationary distribution \(\mathbf{v}\)
as we have already discussed. The long-term payoff of player $2$ is now given
by,

\begin{equation*}\label{eq:PD_long_term_payoff}
  \mathbf{s_{\mathbf{\tilde{p}}, \mathbf{p}}} = a_R \cdot R + a_S \cdot S + a_T \cdot T + a_P \cdot P, ~~where~~
\end{equation*}

\begin{equation*}
  \begin{array}{ll}
  a_R = & \displaystyle \sum_{h \in H^2} u^2_h \cdot p_h \cdot \tilde{p}_h, \\ [0.2cm]
  a_S = & \displaystyle \sum_{h \in H^2} u^2_h \cdot (1 - p_h) \cdot \tilde{p}_h, \\ [0.2cm]
  a_T = & \displaystyle \sum_{h \in H^2} u^2_h \cdot p_h \cdot (1 - \tilde{p}_h), \\ [0.2cm]
  a_P = &  \displaystyle \sum_{h \in H^2} u^2_h \cdot (1 - p_h) \cdot (1 - \tilde{p}_h).
\end{array}
\end{equation*}

The payoff for player $2$ can be expressed analogously.

For the case of reactive-two strategies.

\begin{corollary}\label{corollary:reactive_two_partner_strategies_PD}
A nice reactive-two strategy $\mathbf{p}$, is a partner strategy if and only if,
the strategy entries satisfy the conditions:

\begin{equation*}
  \begin{array}{rl}
    (P - T)\, p_{DD} & < P - R, \\ [0.2cm]
    \quad (T - P) \, (p_{CD} + p_{DC}) + (R - S)\,p_{DD} & < 3 R - S - 2\,P, \\ [0.2cm]
    (T - P)\, p_{CD} + (R - S)\, (p_{CD} + p_{DD}) & < 4\,R - 2\,S - P - T, \\ [0.2cm]
    (R - S)\,(p_{CD} + p_{DC}) & < 3 R - 2 S - T, \\ [0.2cm]
    \left(R - S\right)\, p_{CD} + \left(T - P\right)\,p_{DC} & < 2 R - S - P
\end{array}
\end{equation*}
\end{corollary}

For the case of reactive-three strategies.

\begin{corollary}\label{corollary:reactive_three_partner_strategies_PD}
A nice reactive-three strategy $\mathbf{p}$, is a partner strategy if and only if,
the strategy entries satisfy the conditions:

\begin{equation*}
  \begin{array}{ccc}
    p_{DDD} \left(R - S\right) + \left(T - P\right) \left(p_{CDD} + p_{DCD} + p_{DDC}\right) & < & 4 R - 3 P  - S \\ [0.2cm]
    p_{CDC} \left(T - P\right) + p_{DCD} \left(R - S\right) & < & 2 R - P  - S \\ [0.2cm]
    - P \left(p_{DDD} - 1\right) + T p_{DDD} & < & R \\ [0.2cm] 
    \left(T - P\right) \left(p_{CCD} + p_{CDD} + p_{DDC}\right) + \left(R - S\right) \left(p_{CDC} + p_{DCC} + p_{DCD} + p_{DDD}\right) & < & 8 R - 3 P - 4 S - T \\ [0.2cm]
    p_{DCC} \left(T - P\right) + \left(R - S\right) \left(p_{CCD} + p_{CDC}\right) & < & 3 R - P - 2 S \\ [0.2cm]
    \left(T - P\right) \left(p_{CCD} + p_{DCC} + p_{DDC}\right) + \left(R - S\right) \left(p_{CDC} + p_{CDD} + p_{DCD}\right) & < & 6 R - 3 P - 3 S \\ [0.2cm]
    \left(T - P\right) \left(p_{CCD} + p_{DDC}\right) + \left(R - S\right) \left(p_{CDC} + p_{CDD} + p_{DCC} + p_{DCD}\right) & < & 7 R - 2 P - 4 S - T \\ [0.2cm]
    \left(T - P\right) \left(p_{CCD} + p_{CDD} + p_{DCC}\right) + \left(R - S\right) \left(p_{DDC} + p_{DDD}\right) & < & 5 R - 3 P - 2 S \\ [0.2cm]
    p_{CDD} \left(R - S\right) + \left(T - P\right) \left(p_{DCD} + p_{DDC}\right) & < & 3 R - 2 P - S \\ [0.2cm] 
    p_{CCD} \left(T - P\right) + \left(R - S\right) \left(p_{CDD} + p_{DCC} + p_{DDC}\right) & < & 5 R - P - 3 S - T \\ [0.2cm]
    \left(T - P\right) \left(p_{CCD} + p_{DCC}\right) + \left(R - S\right) \left(p_{CDD} + p_{DDC}\right) & < & 4 R - 2 P - 2 S \\ [0.2cm]
    \left(T - P\right) \left(p_{CDC} + p_{DCD}\right) + \left(R - S\right) \left(p_{CCD} + p_{CDD} + p_{DCC} + p_{DDC}\right) & < & 7 R - 2 P - 4 S - T \\ [0.2cm]
    \left(T - P\right) \left(p_{CDC} + p_{CDD} + p_{DCD}\right) + \left(R - S\right) \left(p_{CCD} + p_{DCC} + p_{DDC} + p_{DDD}\right) & < & 8 R - 3 P - 4 S - T \\ [0.2cm]
    \left(T - P\right) \left(p_{CDC} + p_{DCC} + p_{DCD}\right) + \left(R - S\right) \left(p_{CCD} + p_{CDD} + p_{DDC}\right) & < & 6 R - 3 P - 3 S \\ [0.2cm]
    \left(T - P\right) \left(p_{CCD} + p_{CDD} + p_{DCC} + p_{DDC}\right) + \left(R - S\right) \left(p_{CDC} + p_{DCD} + p_{DDD}\right) & < & 7 R - 4 P - 3 S \\ [0.2cm]
    \left(R - S\right) \left(p_{CCD} + p_{CDC} + p_{DCC}\right) & < & 4 R - 3 S - T \\ [0.2cm]
    \left(T - P\right) \left(p_{CCD} + p_{CDD}\right) + \left(R - S\right) \left(p_{DCC} + p_{DDC} + p_{DDD}\right) & < & 6 R - 2 P - 3 S - T \\ [0.2cm]
    \left(T - P\right) \left(p_{CDC} + p_{CDD} + p_{DCC} + p_{DCD}\right) + \left(R - S\right) \left(p_{CCD} + p_{DDC} + p_{DDD}\right) & < & 7 R - 4 P - 3 S \\ [0.2cm]
    \end{array}
\end{equation*}
\end{corollary}

\appendix

\section{Proofs for Theorem~\ref{theorem:reactive_two_partner_strategies}}\label{appendix:proofs_reactive_two_partner_strategies}

\subsection{Approach based on Akin's Generalised Lemma}

Suppose player $1$ adopts a reactive-two strategy
$\mathbf{p}\!=\!(p_{CC},p_{CD}, p_{DC}, p_{DD})$. Moreover, suppose player~$2$
adopts an arbitrary memory-2 strategy $\mathbf{m}$. Let $\mathbf{v}=(v_h)_{h\in
H}$ be an invariant distribution of the game between the two players.

The cooperation rate of player $2$ given by~\ref{Eq:coplayer_cooperation_expr}
in the case of $n=2$ is given by,

\begin{equation} \label{Eq:rhoq_n2}
\rho_\mathbf{m} := v^2_{CC} + v^2_{CD} = v^2_{CC} + v^2_{DC}.
\end{equation}

We can use this equality to conclude that

\begin{equation} \label{Eq:EqualityV}
v^2_{CD} = v^2_{DC}.
\end{equation}

Moreover the cooperation rate of player $1$ based on Eq.~\ref{Eq:rhop_alln} is given by,

\begin{equation} \label{Eq:rhop_n2}
\begin{array}{lll}
\rho_\mathbf{p} &= &\displaystyle v^2_{CC}\, p_{CC} +  v^2_{CD}\,p_{CD} + v^2_{DC}\, p_{DC} + v^2_{DD}\, p_{DD}\\[0.2cm]
	& =  &v^2_{CC}\, p_{CC} +  v^2_{CD}\,(p_{CD}\!+\!p_{DC}) + v^2_{DD}\, p_{DD}.
\end{array}
\end{equation}

Here, the second equality is due to Eq.~\eqref{Eq:EqualityV}.
 
\begin{proof}
($\Rightarrow$)A reactive-two strategy \(\mathbf{p} = (p_{CC}, p_{CD}, p_{DC},
p_{DD})\) can only be a Nash equilibrium if {\it no} other strategy yields a
larger payoff, in particular neither \text{AllD} nor the \text{Alternator}
strategy must yield a larger payoff, where

$$
\text{AllD}=(0, 0, 0, 0, 0, 0, 0, 0, 0, 0, 0, 0, 0, 0, 0, 0) \text{ and } \text{Alternator}=(0, 0, 1, 1, 0, 0, 1, 1, 0, 0, 1, 1, 0, 0, 1, 1).
$$

Thus, \(\mathbf{p}\) can only form a Nash equilibrium if 

\begin{align*}
\pi(\text{AllD}, \mathbf{p}) \leq b\!-\!c & \quad \text{ and } \quad \pi(\text{Alternator}, \mathbf{p}) \leq b\!-\!c,
\end{align*}

or equivalently, if

\begin{align}\label{Eq:NashConditionDonationGame}
  p_{DD} \leq 1 - \frac{c}{b} & \quad \text{ and } \quad  p_{CD} + p_{DC} \leq 1 + \frac{b\!-\!c}{c}.
\end{align}

($\Leftarrow$) Now, suppose player $2$ has some strategy $\mathbf{m}$ such that $s_{\mathbf{m},
\mathbf{p}} > b\!-\!c$. It follows that

\begin{equation} \label{Eq:InequalityPartner}
\begin{array}{rcl}
0 	& <	&s_{\mathbf{m}, \mathbf{p}}-(b\!-\!c)\\[0.2cm]
	&\stackrel{Eq.~\eqref{Eq:payoff}}{=}	&b\rho_\mathbf{p} - c\rho_\mathbf{m}-(b\!-\!c)\\[0.2cm]
	&\stackrel{Eqs.~\eqref{Eq:rhoq_n2}, \eqref{Eq:rhop_n2}, \eqref{eq:normalization_marginal_distributions}}{=}	&b\,\Big( v^2_{CC} p_{CC} \!+\!  v^2_{CD}(p_{CD}\!+\!p_{DC}) \!+\! v^2_{DD} p_{DD}\Big) 
		- c\,\Big(v^2_{CC} \!+\! v^2_{CD}\Big) - (b\!-\!c)\Big(v^2_{CC} \!+\!  2v^2_{CD} \!+\! v^2_{DD}\Big)\\[0.2cm]
	&=	&v^2_{CC} \,b\,(p_{CC}-1) + v^2_{CD}\Big(b(p_{CD}\!+\!p_{DC})\!+\!c\!-\!2b\Big) + v^2_{DD}\Big(bp_{DD}-(b\!-\!c)\Big).
\end{array}
\end{equation}

Condition~\eqref{Eq:InequalityPartner} can hold only if,

\begin{equation}
  b \, (p_{CD}\!+\!p_{DC})\!+\!c\!-\!2b > 0,~~~~~ b\, p_{DD} - (b\!-\!c) > 0.
\end{equation}

Thus, Eq.~\eqref{Eq:NashConditionDonationGame} reassures that $\mathbf{p}$
is Nash strategy, and given that $p_{CC} = 1$, it is a partner strategy.
\end{proof}

\subsection{Approach based on Self-Reactive Sufficiency Lemma}

Suppose player $1$ adopts a nice reactive-two strategy
$\mathbf{p}\!=\!(1, p_{CD}, p_{DC}, p_{DD})$. For $\mathbf{p}$ to be a Nash
strategy,

\begin{equation}\label{Eq:NashReactive}
  s_{\mathbf{\tilde{p}}, \mathbf{p}} \leq (b - c),
\end{equation}

must hold against all \(\mathbf{\tilde{p}} \in \tilde{P}\), where \(\tilde{P}\) is the
set of all pure self-reactive-two strategies. In the case of $n=2$, the set
contains 16 strategies.

\begin{proof}
Suppose player $1$ plays a nice reactive-two strategy $\mathbf{p} = (1, p_{CD},
p_{DC}, p_{DD})$, and suppose the co-player $2$ plays a pure self-reactive-two
strategy $\mathbf{\tilde{p}}$. The possible payoffs for
$\mathbf{\tilde{p}} \in \{\mathbf{\tilde{p}}^{0}, \dots, \mathbf{\tilde{p}}^{16}\}$
are:

\begin{equation*}\label{Eq:PayoffExpressionsReactiveTwo}
  \begin{array}{lclc}
    s_{\mathbf{\tilde{p}}^{i}, \mathbf{p}} = & b \cdot p_{DD} & ~~for~~ i \in & \{0, 2, 4, 6, 8, 10, 12, 14\} \\ [1em]
    s_{\mathbf{\tilde{p}}^{i}, \mathbf{p}} = & \frac{b \cdot (p_{CD} + p_{DC} + p_{DD})}{3} - \frac{1}{3} \cdot c  &  ~~for~~ i \in & \{1, 9\} \\ [1em]
    s_{\mathbf{\tilde{p}}^{i}, \mathbf{p}} = & \frac{b \cdot (p_{CD} + p_{DC} + p_{DD} + 1)}{4} - \frac{1}{2} \cdot c  & ~~for~~ i \in & \{3\} \\ [1em]
    s_{\mathbf{\tilde{p}}^{i}, \mathbf{p}} = & \frac{b \cdot (p_{CD} + p_{DC})}{2} - \frac{1}{2} \cdot c  & ~~for~~ i \in & \{4, 5, 12, 13\} \\ [1em]
    s_{\mathbf{\tilde{p}}^{i}, \mathbf{p}} = & \frac{b \cdot (p_{CD} + p_{DC} + 1)}{3} - \frac{2}{3} \cdot c  &  ~~for~~ i \in & \{6, 7\}\\ [1em]
    s_{\mathbf{\tilde{p}}^{i}, \mathbf{p}} = & b - c  & ~~for~~ i \in & \{8, 9, 10, 11, 12, 13, 14, 15\}
  \end{array}
\end{equation*}

Setting the payoff expressions of $s_{\mathbf{\tilde{p}}^{i}, \mathbf{p}}$ to
smaller or equal to $(b - c)$ we get the following unique conditions,

\begin{align} 
  p_{DD} & \leq 1 - \frac{c}{b} \label{Eq:Condition2Reactive1} \\
  \frac{p_{CD} + p_{DC}}{2} & \leq 1 - \frac{1}{2}  \cdot \frac{c}{b} \label{Eq:Condition2Reactive2} \\
  \frac{p_{CD} + p_{DC} + p_{DD}}{3} & \leq 1 - \frac{2}{3} \cdot \frac{c}{b} \label{Eq:Condition2Reactive3}
\end{align}

Notice that only conditions \eqref{Eq:Condition2Reactive1} and
\eqref{Eq:Condition2Reactive2} are necessary.

\end{proof}

\section{Proofs for Theorem~\ref{theorem:reactive_three_partner_strategies}}\label{appendix:proofs_reactive_three_partner_strategies}

\subsection{Approach based on Akin's Generalised Lemma}

Suppose player $1$ adopts a reactive-three strategy $\mathbf{p}$, and suppose
player~$2$ adopts an arbitrary memory-three strategy $\mathbf{m}$. Let
$\mathbf{v}=(v_h)_{h\in H}$ be an invariant distribution of the game between the
two players.

The average cooperation rate $\rho_\mathbf{m}$ of player $2$
(Eq.~\ref{Eq:coplayer_cooperation_expr}) for $n=3$ is given by,

\begin{equation} \label{Eq:rhoq_n3}
\rho_\mathbf{m} := v^2_{CCC} + v^2_{CCD} + v^2_{DCC} + v^2_{DCD} = v^2_{CCC} + v^2_{DCC} + v^2_{CDC} + v^2_{DDC} = v^2_{CCC} + v^2_{CCD} + v^2_{CDC} + v^2_{CDD}.
\end{equation}

We can use this equality to conclude that

\begin{align} 
  v^{2}_{CCD} & = v^{2}_{DCC} \label{Eq:Equality1_n3} \\ 
  v^{2}_{DDC} & = v^{2}_{CDD} \label{Eq:Equality2_n3} \\  
  v^{2}_{CCD} + v^{2}_{DCD}  = v^{2}_{CDC} + v^{2}_{DDC} & \Rightarrow 
  v^{2}_{CCD} = v^{2}_{CDC} + v^{2}_{CDD} - v^{2}_{DCD} \label{Eq:Equality3_n3} 
\end{align}

The average cooperation rate of $1$'s (Eq.~\eqref{Eq:rhop_alln}) for $n=3$ is given by,

\begin{equation}\label{Eq:rhop_n3}
  \begin{array}{rcc}
  \rho_\mathbf{p} & = & \displaystyle v^2_{CCC}\, p_{CCC}+ v^2_{CCD}\, p_{CCD} + v^2_{CDC}\, p_{CDC} + v^2_{CDD}\, p_{CDD} + v^2_{DCD}\, p_{DCD} +  \\ [.7em]
  & & + v^2_{DDC}\, p_{DDC} + v^2_{DDD}\, p_{DDD} \\
  & \stackrel{Eqs.~\eqref{Eq:Equality1_n3}, \eqref{Eq:Equality2_n3}}{=} & \displaystyle v^2_{CCC}\, p_{CCC}+ v^2_{CCD}\, (p_{CCD} + p_{DCC}) + v^2_{CDC}\, p_{CDC} + v^2_{CDD}\, (p_{CDD} + p_{DDC}) + \\ [.7em]
  & & v^2_{DCD}\, p_{DCD} + v^2_{DDD}\, p_{DDD}
  \end{array}
\end{equation}
 
\begin{proof}
($\Rightarrow$) A reactive-three strategy \(\mathbf{p}\) can only
be a Nash equilibrium if {\it no} other strategy yields a larger payoff, in
particular neither \text{AllD} nor the following self-reactive-three strategies,

\begin{align*}
\mathbf{\tilde{p}^{15}} & = (0, 0, 0, 0, 1, 1, 1, 1) \\
\mathbf{\tilde{p}^{17}} & = (0, 0, 0, 1, 0, 0, 0, 1) \\
\mathbf{\tilde{p}^{51}} & = (0, 0, 1, 1, 0, 0, 1, 1) \\
\mathbf{\tilde{p}^{119}} & = (0, 1, 1, 1, 0, 1, 1, 1).
\end{align*}

The above strategies are alternating strategies.For instance,
\(\mathbf{\tilde{p}^{15}}\) and \(\mathbf{\tilde{p}^{51}}\) are delayed
alternating strategies. \(\mathbf{\tilde{p}^{15}}\) cooperates if and only if
defected three rounds ago, and \(\mathbf{\tilde{p}^{15}}\) cooperates after
defecting 2 rounds ago. \(\mathbf{\tilde{p}^{17}}\) and
\(\mathbf{\tilde{p}^{119}}\) alternate between cooperating and defecting after
given sequences occur. Namely,
\(\mathbf{\tilde{p}^{17}}\) cooperates after $DD$ sequence has occurred, and
\(\mathbf{\tilde{p}^{119}}\) defects after $CCC$ sequence has occurred.

\(\mathbf{p}\) can only form a Nash equilibrium if

\begin{align*}
\pi(\text{AllD}, \mathbf{p}) \leq b\!-\!c \quad { and } \quad & \pi(\mathbf{\tilde{p}}^{i}, \mathbf{p}) \leq b\!-\!c \text{ for } i \in \{15, 17, 51, 102\}.
\end{align*}

or equivalently, if

\begin{align}\label{Eq:NashConditionDonationGameN3}
  \begin{split}
    \frac{p_{CCD} + p_{CDC} + p_{DCC}}{3} & < 1\!-\! \frac{1}{3} \cdot \frac{c}{b} \\
    \frac{p_{CDD} + p_{DCD} + p_{DDC}}{3} & < 1\!-\! \frac{2}{3} \cdot \frac{c}{b} \\
    p_{DDD} & < 1\!-\! \frac{c}{b} \\
    \frac{p_{CCD} + p_{CDD} + p_{DCC} + p_{DDC}}{4}  & < 1\!-\! \frac{1}{2} \cdot \frac{c}{b} \\
    \frac{p_{CDC} + p_{DCD}}{2} & < 1\!-\! \frac{1}{2} \cdot \frac{c}{b}
  \end{split}
\end{align}

($\Leftarrow$) Now, suppose player $2$ has some strategy $\mathbf{m}$ 
such that $s_{\mathbf{m}, \mathbf{p}} > b\!-\!c$. It follows that

\begin{equation}\label{Eq:InequalityGoodN3}
\begin{array}{rcl}
0 &\le	&s_{\mathbf{m}, \mathbf{p}}-(b\!-\!c)\\[0.2cm]
	&\stackrel{Eq.~\eqref{Eq:payoff}}{=}	&b\rho_\mathbf{p} - c\rho_\mathbf{m}-(b\!-\!c)\\[0.2cm]
	&\stackrel{Eqs.~\eqref{Eq:rhop_n3}, \eqref{eq:normalization_marginal_distributions}}{=}	& 
  b\,\Big( v^2_{CCC} p_{CCC} \!+\!  v^2_{CCD}(p_{CCD}\!+\!p_{DCC}) \!+\! v^2_{CDC} p_{CDC} \!+\! v^2_{DDC}(p_{CDD}\!+\!p_{DDC}) \!+\! v^2_{DCD} p_{DCD} \!+\! v^2_{DDD} p_{DDD}\Big) \\ [0.2cm]
  & & - c\,\Big(v^2_{CCC} \!+\! 2 v^2_{CCD} \!+\! v^2_{DCD}\Big) - (b\!-\!c)\Big(v^2_{CCC} \!+\! 2 v^2_{CCD} \!+\! v^2_{CDC} \!+\! 2 v^2_{DDC} \!+\! v^2_{DCD} \!+\! v^2_{DDD} \Big) \\ [0.4cm]
  & = & b\, v^2_{CCC} (p_{CCC} - 1) + v^2_{CCD} (b\, (p_{CCD} + p_{DCC} - 2)) + v^2_{CDC} (b\, (p_{CDC} - 1) + c) + \\ [0.2cm]
  & & v^2_{CDD} (b\, (p_{CDD} + p_{DDC} - 2) + 2\, c) + v^2_{DCD} (b\, (p _{DCD} - 1)) + v^2_{DDD} (b\, (p _{DDD} - 1) + c) \\ [0.4cm]
  &\stackrel{Eq.~\eqref{Eq:Equality3_n3}}{=}	& b\, v^2_{CCC} (p_{CCC} - 1) + v^2_{DDD} (b\, (p _{DDD} - 1) + c) + v^2_{CDC} (b\, (p_{CCD} + p_{DCC} + p_{CDC}- 3) + c) + \\ [0.2cm]
  & & v^2_{CDD} (b\, (p_{CDD} + p_{DDC} + p_{CCD} + p_{DCC} - 4) + 2\, c) + v^2_{DCD} (b\, (p _{DCD}  - 1) -  b\, (p_{CCD} + p_{DCC}) - 2)
\end{array}
\end{equation}

Condition~\eqref{Eq:InequalityGoodN3} holds only for,

\begin{equation*}
\begin{array}{c}
b\, (p _{DDD} - 1) + c < 0,  \quad b\, (p_{CCD} + p_{DCC} + p_{CDC}- 3) + c \\ [0.4cm]
b\, (p_{CDD} + p_{DDC} + p_{CCD} + p_{DCC} - 4) + 2\, c < 0 \Rightarrow - b\, (p_{CCD} + p_{DCC} - 2) > b\, (p_{CDD} + p_{DDC} - 2) + 2\, c \\ [0.4cm]
b\, (p _{DCD}  - 1) -  b\, (p_{CCD} + p_{DCC}) - 2 < 0 \Rightarrow  b\, (p _{DCD} + p_{CDD} + p_{DDC} - 3) + 2\, c < 0.
\end{array}
\end{equation*}

Thus, conditions Eq.~\eqref{Eq:NashConditionDonationGameN3} reassure that
$\mathbf{p}$ is Nash strategy, and given that $p_{CC} = 1$, it is a partner
strategy.
\end{proof}

\subsection{Approach based on Self-Reactive Sufficiency Lemma}

Consider all the pure self-reactive-three strategies. There is
a total of 256 such strategies. The payoff expression for each of
these strategies against a nice reactive-three strategies can be calculated
explicitly. We use these expressions to obtain the conditions for partner
strategies similar to the previous section.

\begin{proof}
The payoff expressions for a nice reactive-three strategy $\mathbf{p}$ against all
pure self-reactive-three strategies are as follows,

\small{
\begin{equation}\label{Eq:PayoffExpressionsReactiveThree}
\begin{array}{lcll}
  s_{\mathbf{\tilde{p}}^{i}, \mathbf{p}} = & b \; p_{DDD} & ~~for~~ i \in & \{0, 2, 4, 6, \dots, 250, 252, 254\} \\ [0.1cm]
    s_{\mathbf{\tilde{p}}^{i}, \mathbf{p}} = & \frac{b \cdot (p_{CDD} + p_{DCD} + p_{DDC} + p_{DDD})}{4} - \frac{1}{4} \cdot c & ~~for~~ i \in & \{ 1, 9, 33, 41, 65, 73, 97, 105, 129, 137, 161,
    \\ & & &  169, 193, 201, 225, 233\} \\ [0.1cm]
    s_{\mathbf{\tilde{p}}^{i}, \mathbf{p}} = & \frac{b \cdot \left(p_{CCD} + p_{CDD} + p_{DCC} + p_{DDC} + p_{DDD}\right)}{5} - \frac{2}{5} \cdot c & ~~for~~ i \in & \{ 3, 7, 35, 39, 131, 135, 163, 167\} \\ [0.2cm]
    s_{\mathbf{\tilde{p}}^{i}, \mathbf{p}} = & \frac{b \cdot \left(p_{CDC} + p_{DCD}\right)}{2} - \frac{1}{2} \cdot c & ~~for~~ i \in & \{ 4 \!- \!7, 12 \!- \!15, 20 \!- \!23, 28 \!- \!31, 68 \!- \!71,
    \\ & & &  76 \!- \!79, 84 \!- \!87, 92 \!- \!95, 132 \!- \!135, 
    \\ & & & 140 \!- \!143, 148- 151, 156 \!- \!159, 
    \\ & & & 196 \!- \!199, 204 \!- \!207, 212 \!- \!215, 220 \!- \!223\} \\ 
    s_{\mathbf{\tilde{p}}^{i}, \mathbf{p}} = & \frac{b \cdot \left(p_{CCD} + p_{CDD} + p_{DCC} + p_{DDC} + p_{DDD} + 1\right)}{6} - \frac{1}{2} \cdot c & ~~for~~ i \in & \{ 11, 15, 43, 47\} \\ [0.2cm]
    s_{\mathbf{\tilde{p}}^{i}, \mathbf{p}} = & \frac{b \cdot \left(p_{CDD} + p_{DCD} + p_{DDC}\right)}{3} - \frac{1}{3} \cdot c & ~~for~~ i \in & \{16,17,24,25,48,49,56,57,80,81,88,
    \\ & & & 89,112, 113,120,121, 144,145,152,153,
    \\ & & & 176,177,184,185,208,209,216,217,
    \\ & & & 240, 241,248,249\} \\ 
    s_{\mathbf{\tilde{p}}^{i}, \mathbf{p}} = & \frac{b \cdot \left(p_{CCD} + p_{CDD} + p_{DCC} + p_{DDC}\right)}{4} - \frac{1}{2} \cdot c & ~~for~~ i \in & \{ 18, 19, 22, 23, 50, 51, 54, 55, 146, 147,
    \\ & & &  150, 151, 178, 179, 182, 183\} \\ 
    s_{\mathbf{\tilde{p}}^{i}, \mathbf{p}} = & \frac{b \cdot \left(p_{CCD} + p_{CDD} + p_{DCC} + p_{DDC} + 1\right)}{5} - \frac{3}{5} \cdot c & ~~for~~ i \in & \{ 26, 27, 30, 31, 58, 59, 62, 63\} \\ [0.2cm]
    s_{\mathbf{\tilde{p}}^{i}, \mathbf{p}} = & \frac{b \cdot \left(p_{CCD} + p_{CDC} + p_{CDD} + p_{DCC} + p_{DCD} + p_{DDC} + p_{DDD}\right)}{7}  - \frac{3}{7} \cdot c& ~~for~~ i \in & \{ 37, 67, 165, 195\} \\ [0.2cm]
    s_{\mathbf{\tilde{p}}^{i}, \mathbf{p}} = & \frac{b \cdot \left(p_{CCD} + p_{CDC} + p_{CDD} + p_{DCC} + p_{DCD} + p_{DDC} + p_{DDD} + 1\right)}{8} - \frac{1}{2} \cdot c & ~~for~~ i \in & \{ 45, 75\} \\ [0.2cm]
    s_{\mathbf{\tilde{p}}^{i}, \mathbf{p}} = & \frac{b \cdot \left(p_{CCD} + p_{CDC} + p_{CDD} + p_{DCC} + p_{DCD} + p_{DDC}\right)}{6} - \frac{1}{2} \cdot c & ~~for~~ i \in & \{ 52, 53, 82, 83, 180, 181, 210, 211\} \\  [0.2cm]
    s_{\mathbf{\tilde{p}}^{i}, \mathbf{p}} = & \frac{b \cdot \left(p_{CCD} + p_{CDC} + p_{CDD} + p_{DCC} + p_{DCD} + p_{DDC} + 1\right)}{7} - \frac{4}{7} \cdot c & ~~for~~ i \in & \{ 60, 61, 90, 91\} \\ [0.2cm]
    s_{\mathbf{\tilde{p}}^{i}, \mathbf{p}} = & \frac{b \cdot \left(p_{CCD} + p_{CDC} + p_{DCC}\right)}{3} - \frac{2}{3} \cdot c & ~~for~~ i \in & \{ 96\!- \!103, 112\!- \!119, 224\!- \!231, 240\!- \!247\} \\ [0.2cm]
    s_{\mathbf{\tilde{p}}^{i}, \mathbf{p}} = & \frac{b \cdot \left(p_{CCD} + p_{CDC} + p_{DCC} + 1\right)}{4} - \frac{3}{4} \cdot c & ~~for~~ i \in & \{ 104\!-\!111, 120\!- \!127\} \\ [0.2cm]
    s_{\mathbf{\tilde{p}}^{i}, \mathbf{p}} = & (b - c) & ~~for~~ i \in & \{128, 129, 130, \dots, 255\} \\
\end{array}
\end{equation}}

Setting these to smaller or equal than the mutual cooperation payoff $(b - c)$
give the following ten conditions,

\begin{align}
% \begin{array}{c}
  p_{DDD} \leq 1 \!- \!\frac{c}{b},
  \quad \frac{p_{CDC} + p_{DCD}}{2} \leq 1 - \frac{1}{2} \cdot \frac{c}{b}, 
  \quad \frac{p_{CDD} + p_{DCD} + p_{DDC}}{3} \leq 1 - \frac{2}{3} \cdot \frac{c}{b}, \label{eq:NashConditionsN3PartOne} \\[.5em]
  \frac{p_{CCD} + p_{CDC} + p_{DCC}}{3} \leq 1 - \frac{1}{3} \cdot \frac{c}{b},
  \quad \frac{p_{CCD} + p_{CDD} + p_{DCC} + p_{DDC}}{4} \leq 1 - \frac{1}{2}  \cdot \frac{c}{b} \label{eq:NashConditionsN3PartTwo} \\[.5em]
  \frac{p_{CDD} + p_{DCD} + p_{DDC} + p_{DDD}}{4} \leq 1 - \frac{3}{4} \cdot \frac{c}{b}, \\[.5em]
  \quad \frac{p_{CCD} + p_{CDC} + p_{CDD} + p_{DCC} + p_{DCD} + p_{DDC} + p_{DDD}}{7} \leq 1 - \frac{4}{7} \cdot \frac{c}{b}, \\[.5em]
  \frac{p_{CCD} + p_{CDD} + p_{DCC} + p_{DDC} + p_{DDD}}{5} \leq 1 - \frac{3}{5} \cdot \frac{c}{b}, \\[.5em]
  \quad \frac{p_{CCD} + p_{CDC} + p_{CDD} + p_{DCC} + p_{DCD} + p_{DDC}}{6} \leq 1 - \frac{1}{2} \cdot \frac{c}{b}
% \end{array}
\end{align}

Notice that only the conditions of Eq.~\eqref{eq:NashConditionsN3PartOne}
and~\eqref{eq:NashConditionsN3PartTwo} are necessary. The remaining conditions can
be derived from the sums of conditions in Eq.~\eqref{eq:NashConditionsN3PartOne}
and~\eqref{eq:NashConditionsN3PartTwo}.
\end{proof}

\section{Proof of Corollary~\ref{corollary:reactive_counting_partner_strategies}}

To prove corollary~\ref{corollary:reactive_counting_partner_strategies} we need
to introduce some additional notation. We introduce the vector $\mathbf{w} =
(w_i)_{i \in \{0, 1, \dots, n\}}$, where the entry $w_i$ is the probability that
in the long term outcome the co-player cooperates $i$ times.

An element of $\mathbf{w}$ is the sum of one or more of the marginal distribution
$u^2_{h^2}$ for $h^2 \in H^2$. Namely let,

$$
H^2_{i} = \{h^2 : |a_{C}(h^2)| = i \quad \forall \quad h^2 \in H^2\}, \text{ where }
$$

$$
a_{C}(h^2) = \{a^2_{-t} : a^2_{-t} = C \quad \forall \quad a^2_{-t} \in h^2\}.
$$

Then we define $w_i$ as,

$$
w_{i} = {\sum_{h \in H^2_{i}} v_{h}}.
$$

The cooperation rate of the reactive player is given by,

\begin{equation}\label{Eq:player_cooperation_counting}
  \rho_{\mathbf{p}} = \sum_{i=0}^{n} r_{i} \cdot w_{i}.
\end{equation}

The co-player can use any self-reactive-$n$ strategy, and thus the co-player
differentiates between when the last cooperation/defection occurred. However,
we can still express the co-player's cooperation rate as a function of $w_{i}$.
More specifically, the co-player's cooperation rate is,

\begin{equation}\label{Eq:coplayer_cooperation_counting}
  \rho_{\mathbf{\tilde{p}}} = \sum_{i=0}^{n} \frac{i}{n} \cdot w_{i}.
\end{equation}

We will also define a set of $i-$ repeat strategies. Assume the set of self
reactive strategies $A = \{\mathbf{A^{0}}, \mathbf{A^{1}}, \dots,
\mathbf{A^{n}} \}$, where.

The payoff of an alternating self-reactive-$n$ against a counting-reactive$-n$
$\mathbf{r}$ is given by,

\begin{equation}\label{Eq:alternating_strategies_payoff}
  s_{\mathbf{A^{i}}, \mathbf{r}} = b \cdot r_i - \frac{i}{n} \cdot c ~~for~~ i \in [0, n].
\end{equation}

The intuition behind  Eq.~\eqref{Eq:alternating_strategies_payoff} is that in
the long term, the strategies end up in a state where $\mathbf{A^{i}}$ has cooperated $i$ times
in the last $n$ turns. Thus, the co-player will cooperate and provide the benefit
$b$ with a probability $r_i$, while in return, the alternating strategy has
cooperated $\frac{i}{n}$ times and pays the cost.

With this we have all the required tools to prove the following theorem.

\begin{proof}
($\Rightarrow$) As we have already discussed previously, a strategy can only be
a Nash equilibrium if the payoff of the co-player does not exceed $(b - c)$.
Therefore, for p to be a Nash equilibrium against each strategy in set $A$ (for
$i \in [0, n]$),

\begin{align}\label{eq:NashConditionDonationGameCounting}
  s_{\mathbf{A^{i}}, \mathbf{r}} \leq b - c \\ 
  b \cdot r_i - \frac{i}{n} \cdot c \leq   b - c \\
  r_i \leq 1 - \frac{i}{n} \cdot \frac{c}{b}
\end{align}

Now, suppose player $q$ has some strategy $\mathbf{m}$ and player $p$ has a reactive-counting
strategy such that $s_{\mathbf{m}, \mathbf{p}} > b\!-\!c$. It follows that

\begin{equation}
\begin{array}{rcl}
0 	&\le	&s_{\mathbf{m}, \mathbf{p}}-(b\!-\!c)\\[0.2cm]
  &\stackrel{Eq.~\eqref{Eq:payoff}}{=}	&b\rho_\mathbf{p} - c\rho_\mathbf{m}-(b\!-\!c)\\[0.2cm]
  &\stackrel{Eqs.~\eqref{Eq:normalization_counting}, \eqref{Eq:player_cooperation_counting}, \eqref{Eq:coplayer_cooperation_counting}}{=}	& 
  \displaystyle b\; \sum_{k=0}^{n} r_{n - k} \cdot u_{n - k} - c\, \sum_{k=0}^{n} \frac{n - k}{n} \cdot u_{n - k} - (b - c)\; \sum_{k=0}^{n} u_{n - k}\\[0.2cm]
  & & \displaystyle  u_{n} \Big(b\, (r_n - 1) \Big) +  \sum_{k=1}^{n} u_{n - k} \Big(b\; \sum_{k=1}^{n} r_{n - k} - c\, \sum_{k=0}^{n - 1} \frac{n - k}{n} - (b - c)\Big)
\end{array}
\end{equation}

This condition holds only if,

\begin{align}
  \Big(b\; r_{n - k} - c\, \frac{n - k}{n} - (b - c)\Big) & < 0 \Rightarrow \\ 
  b \, (r_{n - k} - 1) + (1 - \frac{n - k}{n})\, c & < 0 \Rightarrow \\ 
  r_{n - k} < 1 - \frac{n}{k} \cdot \frac{c}{b}.
\end{align}

for $k \in [0, n]$. Thus, any counting strategy that satisfies conditions~\eqref{eq:NashConditionDonationGameCounting}
is Nash, and if it is nice, it's also a partner strategy.
\end{proof}

\section{Proofs for Corollaries~\ref{corollary:reactive_two_partner_strategies_PD}
and~\ref{corollary:reactive_three_partner_strategies_PD}}

\subsection{Reactive-Two Partner Strategies}

There are 16 pure-self reactive strategies in $n=2$. We use calculate the
explicit payoff expressions for each pure self-reactive strategy against a nice
reactive-two strategy as given by Eq.~\eqref{eq:PD_long_term_payoff}.
This gives the following payoff expressions:

\begin{equation*}
  \begin{array}{lcl}
  s_{\mathbf{\tilde{p}}^{i}, \mathbf{p}} = & P (1 - p_{DD}) + T p_{DD} & ~~for~~ i \in \{0, 2, 4, 6, 8, 10, 12, 14\} \\ [0.3cm]
  s_{\mathbf{\tilde{p}}^{i}, \mathbf{p}} = & \frac{-P(p_{CD} + p_{DC} - 2) + Rp_{DD} - S(p_{DD} - 1) + T(p_{CD} + p_{DC})}{3} & ~~for~~ i \in \{1, 9\} \\ [0.3cm]
  s_{\mathbf{\tilde{p}}^{i}, \mathbf{p}} = & \frac{P(1 - p_{CD}) + R(p_{DC} + p_{DD}) - S(p_{DC} + p_{DD} - 2) + T(p_{CD} + 1)}{4} & ~~for~~ i \in \{3\} \\ [0.3cm]
  s_{\mathbf{\tilde{p}}^{i}, \mathbf{p}} = & \frac{P(1 - p_{DC}) + Rp_{CD} - S(p_{CD} - 1) + Tp_{DC}}{2} & ~~for~~ i \in \{4, 5, 12, 13\} \\ [0.3cm]
  s_{\mathbf{\tilde{p}}^{i}, \mathbf{p}} = & \frac{R(p_{CD} + p_{DC}) - S(p_{CD} + p_{DC} - 2) + T}{3} & ~~for~~ i \in \{6, 7\} \\ [0.3cm]
  s_{\mathbf{\tilde{p}}^{i}, \mathbf{p}} = & R & ~~for~~ i \in \{8, 9, 10, 11, 12, 13, 14, 15\} \\ [0.3cm]
\end{array}
\end{equation*}

Setting the above expressions to $\leq R$ gives the following conditions,

\begin{equation*}
  \begin{array}{cl}
  (P - T)\, p_{DD} & < P - R, \\ [0.4cm]
  \quad (T - P) \, (p_{CD} + p_{DC}) + (R - S)\,p_{DD} & < 3 R + S - 2\,P, \\[0.4cm]
  (T - P)\, p_{CD} + (R - S)\, (p_{CD} + p_{DC}) & < 4\,R - 2\,S - P - T, \\[0.4cm]
  \qquad \left(R - S\right)\, p_{CD} + \left(T - P\right)\,p_{DC} & < 2 R - S - P, \\[0.4cm]
  (R - S)\,(p_{CD} + p_{DC}) & < 3 R - 2 S - T.
\end{array}
\end{equation*}

\subsection{Reactive-Three Partner Strategies}

Previously as in the previous subsection we calculate the explicit payoff
expressions for each \(\mathbf{\tilde{p}} \in \tilde{P}\) against a nice
reactive-three. The set of pure self-reactive strategies $\tilde{P}$ in $n=3$
contains 256 elements. The expressions for each strategy are given below,

\begin{equation*}
  \begin{array}{lcll}
    s_{\mathbf{\tilde{p}}^{i}, \mathbf{p}} = & \frac{ (T - P)\, \left(p_{CDD} + p_{DCD} + p_{DDC}\right) + 3\,P + (R - S)\, p_{DDD} + S}{4} &~~for~~ i \in & \{1, 9, 33, 41, 65, 73, 97, 105,\\
    & & & 129, 137, 161, 169, 193, 201, \\
    & & & 225, 233\} \\ [0.2cm]
    s_{\mathbf{\tilde{p}}^{i}, \mathbf{p}} = & \frac{ (T - P)\, p_{CDC} + P + (R - S)\, p_{DCD} + S}{2} &~~for~~ i \in & \{ 4 \!- \!7, 12 \!- \!15, 20 \!- \!23,
    \\ & & &  28 \!- \!31, 68 \!- \!71, 76 \!- \!79,
    \\ & & &  84 \!- \!87, 92 \!- \!95, 132 \!- \!135,
    \\ & & & 140 \!- \!143, 148- 151, 156 \!- \!159,
    \\ & & & 196 \!- \!199, 204 \!- \!207, 212 \!- \!215,
    \\ & & & 220 \!- \!223\} \\ [0.2cm]
    s_{\mathbf{\tilde{p}}^{i}, \mathbf{p}} = &- P \left(p_{DDD} - 1\right) + T p_{DDD} &~~for~~ i \in & \{0, 2, 4, \dots, 252, 254\}\\[0.2cm]
    s_{\mathbf{\tilde{p}}^{i}, \mathbf{p}} = & \frac{ (T - P)\, (p_{CCD} + p_{CDD} + p_{DDC}) + 3\,P + (R - S)\, (p_{CDC} + p_{DCC} + p_{DCD} + p_{DDD}) + 4\,S + T}{8} &~~for~~ i \in & \{45\} \\[0.2cm]
    s_{\mathbf{\tilde{p}}^{i}, \mathbf{p}} = & \frac{ (T - P)\, p_{DCC} + P + (R - S)\, (p_{CDC} + p_{CCD}) + 2\,S}{3}  &~~for~~ i \in & \{ 96\!- \!103, 112\!- \!119, 
    \\ & & & 224\!- \!231, 240\!- \!247\} \\[0.2cm]
    s_{\mathbf{\tilde{p}}^{i}, \mathbf{p}} = & \frac{ (T - P)\, (p_{CCD} + p_{DCC} + p_{DDC}) + 3\,P + (R - S)\, (p_{CDC} + p_{CDD} + p_{DCD}) + 3\,S}{6}  &~~for~~ i \in & \{52, 53, 180, 181\} \\ [0.2cm]
    s_{\mathbf{\tilde{p}}^{i}, \mathbf{p}} = & \frac{ (T - P)\, (p_{CCD} + p_{DDC}) + 2\,P + T + (R - S)\, (p_{CDC} + p_{CDD} + p_{DCC} + p_{DCD}) + 4\,S}{7}  &~~for~~ i \in & \{60, 61\} \\ [0.2cm]
    s_{\mathbf{\tilde{p}}^{i}, \mathbf{p}} = & \frac{ (T - P)\, (p_{CCD} + p_{CDD} + p_{DCC}) + 3\,P + (R - S)\, (p_{DDC} + p_{DDD}) + 2\,S}{5}  &~~for~~ i \in &  \{3, 7, 35, 39, 131, 135, 163, 167\} \\ [0.2cm]
    s_{\mathbf{\tilde{p}}^{i}, \mathbf{p}} = & \frac{ (T - P)\, (p_{DCD} + p_{DDC}) + 2\,P + (R - S)\, p_{CDD} + S}{3}  &~~for~~ i \in &  
    \{16,17,24,25,48,49,56,
    \\ & & & 57,80,81,88, 89,112, 113,
    \\ & & & 120,121, 144,145,152,153,
    \\ & & & 176,177,184,185,208,209,
    \\ & & & 216,217, 240, 241,248,249\} \\ [0.2cm]
    s_{\mathbf{\tilde{p}}^{i}, \mathbf{p}} = &R &~~for~~ i \in & \{128, 129, \dots, 255\} \\ [0.2cm]
    s_{\mathbf{\tilde{p}}^{i}, \mathbf{p}} = & \frac{ (T - P)\, p_{CCD} + P + T + (R - S)\, (p_{CDD} + p_{DCC} + p_{DDC}) + 3S}{5} &~~for~~ i \in & \{26, 27, 30, 31, 58, 59, 62, 63\}\\ [0.2cm]
    s_{\mathbf{\tilde{p}}^{i}, \mathbf{p}} = & \frac{ (T - P)\, (p_{CCD} + p_{DCC}) + 2\,P + (R - S)\, (p_{CDD} + p_{DDC}) + 2\,S}{4} &~~for~~ i \in & \{18, 19, 22, 23, 50, 51, 54, 55,
    \\ & & &  146, 147, 150, 151, 178, 179, 
    \\ & & & 182, 183\} \\ [0.2cm]
    s_{\mathbf{\tilde{p}}^{i}, \mathbf{p}} = & \frac{(T - P)\, (p_{CDC} + p_{DCD}) + 2\, P + T + (R - S)\, (p_{CCD} + p_{CDD} + p_{DCC} + p_{DDC}) + 4\,S}{7} &~~for~~ i \in & \{90, 91\} \\ [0.2cm]
    s_{\mathbf{\tilde{p}}^{i}, \mathbf{p}} = & \frac{(T - P)\, (p_{CDC} + p_{CDD} + p_{DCD}) + 3\, P + T + (R - S)\, (p_{CCD} + p_{DCC} + p_{DDC} + p_{DDD}) + 4\,S}{8} &~~for~~ i \in & \{75\} \\ [0.2cm]
    s_{\mathbf{\tilde{p}}^{i}, \mathbf{p}} = & \frac{(T - P)\, (p_{CDC} + p_{DCC} + p_{DCD}) + 3\, P + (R - S)\, (p_{CCD} + p_{CDD} + p_{DDC}) + 3\,S}{6} &~~for~~ i \in & \{82, 83, 210, 211\} \\ [0.2cm]
    s_{\mathbf{\tilde{p}}^{i}, \mathbf{p}} = & \frac{(T - P)\, (p_{CCD} + p_{CDD} + p_{DCC} + p_{DDC}) + 4\, P + (R - S)\, (p_{CDC} + p_{DCD} + p_{DDD}) + 3\,S}{7} &~~for~~ i \in & \{37, 165\}\\ [0.2cm]
    s_{\mathbf{\tilde{p}}^{i}, \mathbf{p}} = & \frac{T  + (R - S)\, (p_{CCD} + p_{CDC} + p_{DCC}) + 3\,S}{4} &~~for~~ i \in & \{ 104\!-\!111, 120\!- \!127\} \\ [0.2cm]
    s_{\mathbf{\tilde{p}}^{i}, \mathbf{p}} = & \frac{(T - P)\, (p_{CCD} + p_{CDD}) + 2\,P + T + (R - S)\, (p_{DCC} + p_{DDC} + p_{DDD}) + 3\,S}{6} &~~for~~ i \in & \{11, 15, 43, 47\}\\ [0.2cm]
    s_{\mathbf{\tilde{p}}^{i}, \mathbf{p}} = & \frac{(T - p)\, (p_{CDC} + p_{CDD} + p_{DCC} + p_{DCD}) + 4\,P + (R - S)\, (p_{CCD} + p_{DDC} + p_{DDD}) + 3\,S}{7} &~~for~~ i \in & \{67, 195\}
\end{array}
\end{equation*}

Setting the above expressions to $\leq R$ gives the following conditions,

\begin{equation*}
  \begin{array}{ccc}
    p_{DDD} \left(R - S\right) + \left(T - P\right) \left(p_{CDD} + p_{DCD} + p_{DDC}\right) & < & 4 R - 3 P  - S \\
    p_{CDC} \left(T - P\right) + p_{DCD} \left(R - S\right) & < & 2 R - P  - S \\
    - P \left(p_{DDD} - 1\right) + T p_{DDD} & < & R \\ 
    \left(T - P\right) \left(p_{CCD} + p_{CDD} + p_{DDC}\right) + \left(R - S\right) \left(p_{CDC} + p_{DCC} + p_{DCD} + p_{DDD}\right) & < & 8 R - 3 P - 4 S - T \\
    p_{DCC} \left(T - P\right) + \left(R - S\right) \left(p_{CCD} + p_{CDC}\right) & < & 3 R - P - 2 S \\
    \left(T - P\right) \left(p_{CCD} + p_{DCC} + p_{DDC}\right) + \left(R - S\right) \left(p_{CDC} + p_{CDD} + p_{DCD}\right) & < & 6 R - 3 P - 3 S \\
    \left(T - P\right) \left(p_{CCD} + p_{DDC}\right) + \left(R - S\right) \left(p_{CDC} + p_{CDD} + p_{DCC} + p_{DCD}\right) & < & 7 R - 2 P - 4 S - T \\
    \left(T - P\right) \left(p_{CCD} + p_{CDD} + p_{DCC}\right) + \left(R - S\right) \left(p_{DDC} + p_{DDD}\right) & < & 5 R - 3 P - 2 S \\
    p_{CDD} \left(R - S\right) + \left(T - P\right) \left(p_{DCD} + p_{DDC}\right) & < & 3 R - 2 P - S \\ 
    p_{CCD} \left(T - P\right) + \left(R - S\right) \left(p_{CDD} + p_{DCC} + p_{DDC}\right) & < & 5 R - P - 3 S - T \\
    \left(T - P\right) \left(p_{CCD} + p_{DCC}\right) + \left(R - S\right) \left(p_{CDD} + p_{DDC}\right) & < & 4 R - 2 P - 2 S \\
    \left(T - P\right) \left(p_{CDC} + p_{DCD}\right) + \left(R - S\right) \left(p_{CCD} + p_{CDD} + p_{DCC} + p_{DDC}\right) & < & 7 R - 2 P - 4 S - T \\
    \left(T - P\right) \left(p_{CDC} + p_{CDD} + p_{DCD}\right) + \left(R - S\right) \left(p_{CCD} + p_{DCC} + p_{DDC} + p_{DDD}\right) & < & 8 R - 3 P - 4 S - T \\
    \left(T - P\right) \left(p_{CDC} + p_{DCC} + p_{DCD}\right) + \left(R - S\right) \left(p_{CCD} + p_{CDD} + p_{DDC}\right) & < & 6 R - 3 P - 3 S \\
    \left(T - P\right) \left(p_{CCD} + p_{CDD} + p_{DCC} + p_{DDC}\right) + \left(R - S\right) \left(p_{CDC} + p_{DCD} + p_{DDD}\right) & < & 7 R - 4 P - 3 S \\
    \left(R - S\right) \left(p_{CCD} + p_{CDC} + p_{DCC}\right) & < & 4 R - 3 S - T \\
    \left(T - P\right) \left(p_{CCD} + p_{CDD}\right) + \left(R - S\right) \left(p_{DCC} + p_{DDC} + p_{DDD}\right) & < & 6 R - 2 P - 3 S - T \\
    \left(T - P\right) \left(p_{CDC} + p_{CDD} + p_{DCC} + p_{DCD}\right) + \left(R - S\right) \left(p_{CCD} + p_{DDC} + p_{DDD}\right) & < & 7 R - 4 P - 3 S \\
    \end{array}
\end{equation*}

~\\
\bibliography{bibliography.bib}

\end{document}