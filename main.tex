\documentclass{article}

\usepackage{minitoc}
\usepackage{tabularx}
\usepackage{booktabs}
\usepackage{amssymb}
\usepackage{graphics}
\usepackage{hyperref}
\usepackage{xcolor}
\usepackage{amsmath}
\usepackage{blkarray}

\usepackage[margin=2.5cm, includefoot, footskip=30pt]{geometry}
\pagestyle{plain}
\setlength{\parindent}{0em}
\setlength{\parskip}{1em}
\renewcommand{\baselinestretch}{1}

\newcommand{\nikoleta}[1]{\textcolor{orange}{{\bf NG:} #1}}

\title{$N-$bits reactive strategies in repeated games}

\author{Nikoleta E. Glynatsi, Christian Hilbe, Martin Nowak}
\date{}

\begin{document}

\maketitle

\section{Introduction}

In this work we explore \textit{reactive strategies} in the infinitely repeated
donation game. The donation is a (\(2 \times 2\)) symmetric game where at each
turn players \(p\) and \(q\), simultaneously and independently, decide to
cooperate (C) or to defect (D). Thus, there are four outcomes in each single
round, \(xy \in \{CC, CD, DC, DD\}\), where \(x\) and \(y\) represent \(p\)'s
and \(q\)'s choices respectively.

Each then receives a payoff. The following (\(2 \times 2\)) payoff matrix describes
the payoffs of both players in each round,

\begin{equation}\label{eq:payoffs}
    \begin{blockarray}{ccc}
        & \text{cooperate} & \text{defect} \\
        \begin{block}{c(cc)}
            \text{cooperate} & b - c & -c \\
            \text{defect} & b & 0 \\
        \end{block}
    \end{blockarray}
  \end{equation}

where \(b > c\). Alternatively, we can define the payoff vectors for each player by

\begin{equation}\label{eq:vector_payoffs}
  S_{p} = (b-c, -c, b, 0) \quad \textrm{and} \quad  S_{q} = (b-c, b, -c, 0).
\end{equation}

\textit{Reactive} strategies are strategies that take into account the actions
of the co-player to make a decision in each turn of the repeated game. Reactive
strategies are studied in the literature due to their mathematical tractability.
A play between two reactive strategies can be described as a Markov process with
a transition matrix \(M\). The expected payoffs to \(p\) and \(q\), can then be
explicitly calculated using the the stationary distribution \(\mathbf{v}\) of
\(M\) and the respective round payoffs \(S_{p} \textrm{ and } S_{q}\).

\section{Results}

\subsection{1-bit reactive}

The literature has extensively studied reactive strategies that take into
account only the last turn of the opponent. Here will reefer to these as
\textit{one-bit reactive strategies}. One-bit reactive strategies can be written
as a 2-tuple \(p = (p_{C}, p_{D})\) where \(p_{C}\) is the probability of
cooperating after the co-player has cooperated and \(p_{D}\) after they
defected.

The play of reactive strategies can be modelled as a Markov chain. In the case
of the one-bit reactive strategies, there are only 4 possibles states \(\{CC,
CD, DC, DD\}\) and the transition matrix is given by,

\begin{equation}
\displaystyle M_1 = \left[\begin{matrix}p_{1} q_{1} & p_{1} \left(1 - q_{1}\right) & q_{1} \left(1 - p_{1}\right) & \left(1 - p_{1}\right) \left(1 - q_{1}\right)\\p_{2} q_{1} & p_{2} \left(1 - q_{1}\right) & q_{1} \left(1 - p_{2}\right) & \left(1 - p_{2}\right) \left(1 - q_{1}\right)\\p_{1} q_{2} & p_{1} \left(1 - q_{2}\right) & q_{2} \left(1 - p_{1}\right) & \left(1 - p_{1}\right) \left(1 - q_{2}\right)\\p_{2} q_{2} & p_{2} \left(1 - q_{2}\right) & q_{2} \left(1 - p_{2}\right) & \left(1 - p_{2}\right) \left(1 - q_{2}\right)\end{matrix}\right].
\end{equation}

A probability distribution \(\mathbf{v}^1\) on the set of outcomes is a
non-negative vector with unit sum, indexed by the four states for which,

\begin{equation*}
  \mathbf{v}^{1} M^{1} = \mathbf{v}^{1}.
\end{equation*}

With respect to \( \mathbf{v}^{1}\) the expected payoffs to \(p\) and \(q\),
denoted \(\pi_{(p, q)}\) and \(\pi_{(q, p)}\), are the dot products with the
corresponding payoff vectors:

\begin{equation}
  \pi_{(p, q)} = \mathbf{v}^{1} \cdot S_{p} \quad \textrm{and} \quad \pi_{(q, p)} = \mathbf{v}^{1} \cdot S_{q}.
\end{equation}

In the case of the one-bit reactive strategies the payoffs' analytical
expressions are tractable. For example,

\begin{equation}
  \pi_{(p, q)} = \frac{c (\text{p1} \text{q2}+\text{p2} (-\text{q2})+\text{p2})-b (\text{p2} (\text{q1}-\text{q2})+\text{q2})}{(\text{p1}-\text{p2}) (\text{q1}-\text{q2})-1}
\end{equation}

\subsection{2-bits reactive}

In the case of \textit{two-bit reactive strategies}, strategies are based on the
actions of the co-player in the previous two rounds. Since for a single round
there are 4 possible outcomes, for two rounds there will be \(4 \times 4 = 16\)
possible situations.

We denote the states as \(E_x E_y | F_x F_y\) where the outcome of the previous
round is \(E_x E_y\) and the outcome of the current round is \(F_x F_y\) and
\(E_x, E_y, F_x, F_y \in \{C, D\}\) then the state space is \(\{CC|CC, CC|CD,
CC|DC, CC|DD, \allowbreak CD|CC, CD|CD, CD|DC, CD|DD, DC|CC, DC|CD, DC|DC,
DC|DD, DD|CC, DD|CD,DD|DC,DD|DD\}\). Thus, \(p = (p_1, p_2, \dots, p_{16})\)
corresponding to the previous state \(E_x E_y | F_x F_y \in
\{CC|CC,CC|CD,CC|DC,\dots, \allowbreak DD|CD,DD|DC,DD|DD\}\). Similarly, \(q =
(q_1, q_2, \dots, q_{16})\). Following the same approach as the one-bit case, a
play between two two-bits reactive strategies can be described by a Markov
process where now the states are given by \(E_x, E_y, F_x, F_y \in \{C, D\}\),
and the stationary distribution \(\mathbf{v}^{2} \in R^{16}_{[0, 1]}\) is the
solution to \( \mathbf{v}^{2} M^{2} = \mathbf{v}^{2}\). The transition matrix
\(M^2\) is a \((16 \times 16)\) matrix where \(M(v^{2}_{n + 1} = G_x G_y | H_x
H_y | v^{2}_{n}= E_x E_y | F_x F_y = 0)\) if \(G_x G_y \neq F_x F_y\), so in
each row of the matrix there will be at most four nonzero elements.
Thus,

\resizebox{\linewidth}{!}{%
$
M^{2} = \left(\begin{array}{cccccccccccccccc}
 \text{p1} \text{q1} & \text{p1} (1-\text{q1}) & (1-\text{p1}) \text{q1} & (1-\text{p1}) (1-\text{q1}) & 0 & 0 & 0 & 0 & 0 & 0 & 0 & 0 & 0 & 0 & 0 & 0 \\
 0 & 0 & 0 & 0 & \text{p2} \text{q1} & \text{p2} (1-\text{q1}) & (1-\text{p2}) \text{q1} & (1-\text{p2}) (1-\text{q1}) & 0 & 0 & 0 & 0 & 0 & 0 & 0 & 0 \\
 0 & 0 & 0 & 0 & 0 & 0 & 0 & 0 & \text{p1} \text{q2} & \text{p1} (1-\text{q2}) & (1-\text{p1}) \text{q2} & (1-\text{p1}) (1-\text{q2}) & 0 & 0 & 0 & 0 \\
 0 & 0 & 0 & 0 & 0 & 0 & 0 & 0 & 0 & 0 & 0 & 0 & \text{p2} \text{q2} & \text{p2} (1-\text{q2}) & (1-\text{p2}) \text{q2} & (1-\text{p2}) (1-\text{q2}) \\
 \text{p3} \text{q1} & \text{p3} (1-\text{q1}) & (1-\text{p3}) \text{q1} & (1-\text{p3}) (1-\text{q1}) & 0 & 0 & 0 & 0 & 0 & 0 & 0 & 0 & 0 & 0 & 0 & 0 \\
 0 & 0 & 0 & 0 & \text{p4} \text{q1} & \text{p4} (1-\text{q1}) & (1-\text{p4}) \text{q1} & (1-\text{p4}) (1-\text{q1}) & 0 & 0 & 0 & 0 & 0 & 0 & 0 & 0 \\
 0 & 0 & 0 & 0 & 0 & 0 & 0 & 0 & \text{p3} \text{q2} & \text{p3} (1-\text{q2}) & (1-\text{p3}) \text{q2} & (1-\text{p3}) (1-\text{q2}) & 0 & 0 & 0 & 0 \\
 0 & 0 & 0 & 0 & 0 & 0 & 0 & 0 & 0 & 0 & 0 & 0 & \text{p4} \text{q2} & \text{p4} (1-\text{q2}) & (1-\text{p4}) \text{q2} & (1-\text{p4}) (1-\text{q2}) \\
 \text{p1} \text{q3} & \text{p1} (1-\text{q3}) & (1-\text{p1}) \text{q3} & (1-\text{p1}) (1-\text{q3}) & 0 & 0 & 0 & 0 & 0 & 0 & 0 & 0 & 0 & 0 & 0 & 0 \\
 0 & 0 & 0 & 0 & \text{p2} \text{q3} & \text{p2} (1-\text{q3}) & (1-\text{p2}) \text{q3} & (1-\text{p2}) (1-\text{q3}) & 0 & 0 & 0 & 0 & 0 & 0 & 0 & 0 \\
 0 & 0 & 0 & 0 & 0 & 0 & 0 & 0 & \text{p1} \text{q4} & \text{p1} (1-\text{q4}) & (1-\text{p1}) \text{q4} & (1-\text{p1}) (1-\text{q4}) & 0 & 0 & 0 & 0 \\
 0 & 0 & 0 & 0 & 0 & 0 & 0 & 0 & 0 & 0 & 0 & 0 & \text{p2} \text{q4} & \text{p2} (1-\text{q4}) & (1-\text{p2}) \text{q4} & (1-\text{p2}) (1-\text{q4}) \\
 \text{p3} \text{q3} & \text{p3} (1-\text{q3}) & (1-\text{p3}) \text{q3} & (1-\text{p3}) (1-\text{q3}) & 0 & 0 & 0 & 0 & 0 & 0 & 0 & 0 & 0 & 0 & 0 & 0 \\
 0 & 0 & 0 & 0 & \text{p4} \text{q3} & \text{p4} (1-\text{q3}) & (1-\text{p4}) \text{q3} & (1-\text{p4}) (1-\text{q3}) & 0 & 0 & 0 & 0 & 0 & 0 & 0 & 0 \\
 0 & 0 & 0 & 0 & 0 & 0 & 0 & 0 & \text{p3} \text{q4} & \text{p3} (1-\text{q4}) & (1-\text{p3}) \text{q4} & (1-\text{p3}) (1-\text{q4}) & 0 & 0 & 0 & 0 \\
 0 & 0 & 0 & 0 & 0 & 0 & 0 & 0 & 0 & 0 & 0 & 0 & \text{p4} \text{q4} & \text{p4} (1-\text{q4}) & (1-\text{p4}) \text{q4} & (1-\text{p4}) (1-\text{q4}) \\
\end{array}\right).$}

Computing the stationary distribution of \(M^{2}\) analytically is not possible.
Thus, computing the payoffs \(\pi\) for two generic two-bits reactive strategies
is not possible either.

However, some expressions are still obtainable.

\subsubsection{2-bits reactive strategies against ALLC and ALLD}

For \(p = (p_1, p_2, \dots, p_{16})\) and \(q=(0, 0, \dots, 0)\) (ALLD),

\[\pi_{(p, \text{ALLD})} = - c p_4.\]

For \(p = (p_1, p_2, \dots, p_{16})\) and \(q=(1, 1, \dots, 1)\) (ALLC),

\[\pi_{(p, \text{ALLC})} = b - cp_1.\]
 

\subsubsection{2-bits deterministic reactive strategies.}

There are a total of (\(4 ^ 2\)) \(16\) two-bit deterministic reactive
strategies, Table~\ref{table:two_bits_strategies}.

\begin{table}[h!]
  \centering
   \begin{tabular}{c c c c c}
  \toprule
   $p_1$ & $p_2$ & $p_3$ & $p_4$ & name \\ [0.5ex] 
   \toprule
   0 & 0 & 0 & 0 & ALLD \\
   0 & 0 & 0 & 1 & S1 \\
   0 & 0 & 1 & 0 & S2 \\
   0 & 0 & 1 & 1 & S3 \\
   0 & 1 & 0 & 0 & S4 \\
   0 & 1 & 0 & 1 & S5 \\
   0 & 1 & 1 & 0 & S6 \\
   0 & 1 & 1 & 1 & S7 \\
   1 & 0 & 0 & 0 & S8 \\
   1 & 0 & 0 & 1 & S9 \\
   1 & 0 & 1 & 0 & S10 \\
   1 & 0 & 1 & 1 & S11 \\
   1 & 1 & 0 & 0 & S12 \\
   1 & 1 & 0 & 1 & S13 \\
   1 & 1 & 1 & 0 & S14 \\
   1 & 1 & 1 & 1 & ALLC \\
   \hline
   \end{tabular}
   \caption{\textbf{Deterministic two-bits reactive strategies}}\label{table:two_bits_strategies}.
\end{table}

\input{deterministic_payoffs.txt}

\input{deterministic_payoffs_co_player.txt}

Using the formulation of (\ref{}) we can numerically compute the payoffs without
simulations.


\section{Numerical Results}

We use an evolutionary process based on on Nowak and Imphof.

\begin{figure}
  \includegraphics{}
\end{figure}


\end{document}