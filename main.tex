\documentclass{article}

\usepackage{minitoc}
\usepackage{tabularx}
\usepackage{booktabs}
\usepackage{amssymb}
\usepackage{graphics}
\usepackage{hyperref}
\usepackage{xcolor}
\usepackage{amsmath}

\usepackage[margin=2.5cm, includefoot, footskip=30pt]{geometry}

\newcommand{\nikoleta}[1]{\textcolor{orange}{{\bf NG:} #1}}

\title{Two bits (and more) reactive strategies in repeated games}

\author{Nikoleta E. Glynatsi, Christian Hilbe, Martin Nowak}
\date{}

\begin{document}

We explore infinetly () symmetric games repeated games where players use
the family reactive strategies to make decision at each turn. In the donation
game at each turn each player can, simulanetlous and idependanlty, choose
to cooperate or defect. The payoffs are given by:


Strategies of the reactive family only take into account the actions of the co-player.
Famously one bit reactive strategies are given by p. Here we exlore the case.
Here we explore the cases where. 

The players can be in 


The transition matrix given by: 

Computing the stationary distribution of this matrix (analytically) is not
trivia.

\subsection{Analytical analysis}


\begin{align}
  y &= \begin{bmatrix}
         x_{1} \\
         x_{2} \\
         \vdots \\
         x_{m}
       \end{bmatrix}
\end{align}


\begin{align}
[   &  p_{1}^{2} q_{1}^{2} \\
    &  p_{1}^{2} q_{1} \left(1 -q_{1}\right) \\
    &  p_{1} q_{1}^{2} \left(1 - p_{1}\right) \\ 
    &  p_{1} q_{1} \left(p_{1} -1\right) \left(q_{1} - 1\right) \\
    &  p_{1}^{2} q_{1} \left(1 -q_{1}\right) \\
    &  p_{1}^{2} \left(q_{1} - 1\right)^{2}\\
    &  p_{1} q_{1} \left(p_{1} - 1\right) \left(q_{1} - 1\right)\\
    &  - p_{1} \left(p_{1} - 1\right) \left(q_{1} - 1\right)^{2}\\
    & p_{1} q_{1}^{2} \left(1 - p_{1}\right)\\
    & p_{1} q_{1} \left(p_{1} - 1\right) \left(q_{1} - 1\right)\\
    & q_{1}^{2} \left(p_{1} - 1\right)^{2} \\
    & - q_{1}\left(p_{1} - 1\right)^{2} \left(q_{1} - 1\right)\\
    & p_{1} q_{1} \left(p_{1} - 1\right) \left(q_{1} - 1\right)\\
    & - p_{1} \left(p_{1} - 1\right) \left(q_{1} - 1\right)^{2}\\
    & - q_{1} \left(p_{1} - 1\right)^{2} \left(q_{1} - 1\right)\\
    & \left(p_{1} - 1\right)^{2} \left(q_{1} - 1\right)^{2}]
\end{align}


\subsection{Numerical analysis}

For proof that our formulation is correct to the jupyter notebppk `Numerical
simulanetlous'.

An evolutionary approach based on Nowak and Imphof gives the following
results when we vary the benefit \(c\).


\end{document}