\documentclass{article}

\usepackage{minitoc}
\usepackage{tabularx}
\usepackage{booktabs}
\usepackage{graphicx}
\usepackage{hyperref}
\usepackage{xcolor}
\usepackage{blkarray}
\usepackage{amsthm, amssymb, amsmath}
\usepackage{caption}
\usepackage{subcaption}
\usepackage{multirow}
\usepackage[ruled,vlined]{algorithm2e}

\usepackage{natbib}
\bibliographystyle{abbrvnat}

\theoremstyle{definition}
\newtheorem{definition}{Definition}[section]
\newtheorem{theorem}{Theorem}[section]
\newtheorem{lemma}[theorem]{Lemma}
\newtheorem{conjecture}[theorem]{Conjecture}

\usepackage[margin=2.5cm, includefoot, footskip=30pt]{geometry}
\pagestyle{plain}
\setlength{\parindent}{0em}
\setlength{\parskip}{1em}

\renewcommand{\baselinestretch}{1}

\usepackage{standalone}

\newtheorem{proposition}{Proposition}

\title{$n-$bits reactive strategies}

\author{Nikoleta E. Glynatsi, Ethan Akin, Martin Nowak, Christian Hilbe}
\date{}

\begin{document}

\maketitle


\section{Introduction}

In this work we explore \textit{reactive strategies} in the infinitely repeated
prisoner's dilemma. The prisoner's dilemma is a two person symmetric game that
provides a simple model of cooperation. Each of the two players, \(p\) and
\(q\), simultaneously and independently decide to cooperate (\(C\)) or to defect
(\(D\)). A player who cooperates pays a cost \(c > 0\) to provide a benefit \(b
> c\) for the co-player. A cooperator either gets \(b\!-\!c\) (if the co-player
also cooperates) or \(-c\) (if the co-player defects). Respectively, a defector
either gets \(b\) (if the co-player cooperates) or 0 (if the co-player defects),
and so, the payoffs of player \(p\) take the form,

\begin{equation}\label{eq:donation_payoffs}
  \begin{blockarray}{ccc}
      & \text{cooperate} & \text{defect} \\
      \begin{block}{c(cc)}
          \text{cooperate} & b\!-\!c & -c \\
          \text{defect} & b & 0 \\
      \end{block}
  \end{blockarray}
\end{equation}

The transpose of (\ref{eq:donation_payoffs}) gives the payoffs of co-player
\(q\). We can also define each player's payoffs as vectors,

\begin{equation}\label{eq:vector_payoffs}
  \mathbf{S}_{p} = (b\!-\!c, -c, b, 0) \quad \textrm{and} \quad  \mathbf{S}_{q} = (b\!-\!c, b, -c, 0).
\end{equation}

We denote the long-term payoffs of players \(p\) and \(q\) as \(\mathbf{s}_{p}\)
and \(\mathbf{s}_{q}\).

\section{Model}

At each round \(t\) of the repeated game, players \(p\) and \(q\) decide on an
action \(a^{p}_{t},\) and \(a^{q}_{t} \in \{C, D\}\) respectively (\textbf{Fig.
1a}). We assume that the players' decisions only depend on the outcome of the
previous \(n\) rounds. Am {\it $n$-history for player $p$} is a string
$h^p=(a^p_{-1},\ldots,a^p_{-n})\!\in\!\{C,D\}^n$. Here, an entry $a^p_{-k}$
corresponds to player $p$'s action $k$ rounds ago. Let $H^p$ denote the space of
all $n$-histories of player~$p$. Analogously, we define $H^q$ as the set of
$n$-histories $h^q$ of player~$q$. Sets $H^p$ and $H^q$ contain
$|H^p|=|H^q|=2^{n}$ elements each. A pair $h\!=\!(h^p,h^q)$ is called an {\it
$n$-history of the game}. We use $H=H^p\times H^q$ to denote the space of all
such histories. This set contains $|H|=2^{2n}$ elements.

A {\it memory-$n$} strategy is a vector $\mathbf{p}=(p_h)_{h\in
H}\in[0,1]^{2n}$. Each entry $p_h$ corresponds to the player's cooperation
probability in the next round, depending on the outcome of the previous $n$
rounds.

On the other hand, an {\it $n-$ bit reactive strategy} is a vector
$\mathbf{\hat{p}}=(\hat{p}_h)_{h\in H^q}\in[0,1]^{2n}$. Each entry $\hat{p}_h$
corresponds to the player's cooperation probability in the next round,
depending on the co-player's action(s) of the previous \(n\) rounds. Thus,
\(n\)-bit reactive strategies only depend on the co-player's \(n\)-history
(independent of the focal player's own actions during the past \(n\) rounds).

\begin{figure}[h!]
    \centering
    \includegraphics[width=\textwidth]{figures/conceptual_figure_model.pdf}
    \caption{\textbf{Model.} \textbf{a)} At each turn $t$ of the repeated game,
    players \(p\) and \(q\) decide on an action \(a^{p}_{t}, a^{q}_{t} \in \{C,
    D\}\), respectively. \textbf{b)} Memory$-1$ strategies are a set of very well
    studied strategies that use the actions of both player at round $t-1$ to
    decide on round $t$. In the graphical representation of memory-$1$ strategy
    we demonstrate this by using a single full shaded square. \textbf{c)} In this work we
    focus on reactive strategies, strategies that consider only the
    co-players actions $h^q$. This is demonstrated in the graphical representation
    by the shaded bottom half of the square. \textbf{e)} Let the case of $n=1$, a $1-$bit
    reactive strategy is a vector $\mathbf{\hat{p}}=(\hat{p}_{C}, \hat{p}_{D})$. $\hat{p}_C$
    is the probability of cooperating given that the co-player cooperated and $\hat{p}_D$
    given that they defected. Consider a 
    match between two $1-$bit reactive strategies as shown in panel \textbf{(e)}. The top
    player (player $\hat{p}$) cooperates with a probability $\hat{p}_C$ in the
    second round since the co-player cooperated in the first round, player
    $\hat{p}$ cooperates with a probability $\hat{p}_D$ in the second round
    since the co-player cooperated. The player defects in round 3 with a probability $1 -
    \hat{p}_C$ given that the co-player cooperated. \textbf{d)} Another strategies set that
    we consider is that of $n-$bit self reactive strategies. These are strategies that
    consider only their own previous $n$ action.}
\end{figure}

We say $\!h^q\!=\!(C,\ldots,C)$ is the full cooperation history. An $n-$bit
reactive strategy $\mathbf{\hat{p}}$ is called {\it agreeable} if it prescribes to
cooperate with probability 1 after the full cooperation history.

The strategy $\mathbf{p}$ is called a {\it partner strategy} if it is agreeable
and if expected payoffs satisfy

\begin{equation} \label{Eq:partner}
    s_{\mathbf{q}} \geq b\!-\!c \qquad \Rightarrow \qquad s_{\mathbf{q}} = s_{\mathbf{p}} =  b\!-\!c,
\end{equation}

Thus, if a player uses a partner strategy, both players can share the rewards
fairly. However, if a co-player prefers an unfair approach, they will receive a
reduced payoff as a consequence. Partner strategies, by definition, are best
responses to themselves, making them Nash equilibria~\cite{Hilbe:GEB:2015}. We
wish to characterise all partner $n-bit$ reactive strategies of the repeated
donation game.

\section{Results}

\subsection{Sufficiency of self reactive strategies}

To characterize all partner $n$-bit reactive strategies, one would usually
need to check against all pure $n$-memory one strategies~\cite{mcavoy:PRSA:2019}.
However, we demonstrate that when player $p$ employs an $n$-bit reactive strategy,
it is sufficient to check only against n-bit self-reactive strategies. This
finding aligns with the previous result by Press and Dyson~\cite{press:PNAS:2012}.

More specifically, the result states that for any memory-n strategy used by
player $q$, player $p$s' score is exactly the same as if $q$ had played a
specific self-reactive memory-$n$ strategy.

An maybe example will consider the reactive $\mathbf{\hat{p}} = (0, 1)$ and the
memory-1 strategy Pavlov or Win Stay Lose Shift $\mathbf{p} = (1, 0, 0, 1)$.

\subsection{$2-$bit partner strategies}

For $n=2$, $\mathbf{\hat{p}}=(\hat{p}_{CC}, \hat{p}_{CD}, \hat{p}_{DC})$, where
$p\hat{p}_{CC}$ is the probability of cooperating in round \(t\) when the
co-player cooperated in the last 2 rounds, $\hat{p}_{CD}$ is the probability of
cooperating given that the co-player cooperated in the second to last round and
defected in the last, and so on. An agreeable 2-bit strategy is represented by
the vector $\mathbf{\hat{p}}=(1, \hat{p}_{CD}, \hat{p}_{DC}, \hat{p}_{DD})$:

An agreeable $2-$bit reactive strategy is a partner strategy if the entries of
$\mathbf{\hat{p}}$ satisfy:

\begin{equation}\label{eq:two_bit_conditions}
  \displaystyle \hat{p}_{DD} < 1\!-\! \frac{c}{b}  ~~and~~ \displaystyle \frac{\hat{p}_{CD} + \hat{p}_{DC}}{2} < 1-\frac{c}{2b}.
\end{equation}

A special case of $2-$bit reactive strategies is the {\it $2-$bit counting
reactive} strategies. These are strategies that respond to the action of the
co-player, but they do not differentiate between when defection occurs, only if
one or two defections occurred. Let \(r_i\) be the probability of cooperating
given that the co-player cooperated \(i\) number of times in the last 2 turns.

Thus, $r_2 = \hat{p}_1, r_1 = \hat{p}_2 =  \hat{p}_3, r_0 = \hat{p}_4$ and
$\mathbf{\hat{p}}=(r_0, r_1, r_2)$. Conditions~(\ref{eq:two_bit_conditions})
then become:

\begin{equation}\label{eq:counting_two_bit_conditions}
  \displaystyle r_2 < 1, \quad r_2 < 1-\frac{c}{2b} ~~and~~ r_0 < 1\!-\! \frac{c}{b}.
\end{equation}

\subsection{$3-$bit partner strategies}

For $n=3$, $\mathbf{\hat{p}}=(\hat{p}_{CCC}, \hat{p}_{CCD}, \hat{p}_{CDC},
\hat{p}_{CDD}, \hat{p}_{DCC}, \hat{p}_{DCD}, \hat{p}_{DDC}, \hat{p}_{DDD})$
where $\hat{p}_{CCC}$ is the probability of cooperating in round $t$ when the
co-player cooperates in the last 3rounds, $\hat{p}_{CCD}$ is the probability of
cooperating given that the co-player cooperated in the third and second to last
rounds and defected in the last, etc. An agreeable $3-$bit strategy is of the
vector $\mathbf{\hat{p}}=(1, \hat{p}_{CCD}, \hat{p}_{CDC}, \hat{p}_{CDD},
\hat{p}_{DCC}, \hat{p}_{DCD}, \hat{p}_{DDC}, \hat{p}_{DDD})$.

An agreeable $3-$bit reactive strategy is a partner strategy if the entries of
$\mathbf{\hat{p}}$ satisfy:

\begin{align}\label{eq:three_bit_conditions}
  \hat{p}_{CCD} + \hat{p}_{CDC} + \hat{p}_{DCC} < 3\!-\! \frac{c}{b} & \qquad 
  \hat{p}_{CDD} + \hat{p}_{DCD} + \hat{p}_{DDC} < 3\!-\! \frac{2c}{b} & \qquad 
  \hat{p}_{DDD} < 1\!-\! \frac{c}{b} \\
  & \hat{p}_{CCD} + \hat{p}_{CDD} + \hat{p}_{DDC} +  \hat{p}_{DCD}  < 4\!-\! \frac{2c}{b} 
  & \qquad \hat{p}_{CDC} + \hat{p}_{DCD} < 2\!-\! \frac{c}{b}
\end{align}

A special case of $3-$bit reactive strategies are the $3-$bit counting reactive
strategies. Let $r_i$ be the probability of cooperating given that the co-player
cooperated $i$ number of times in the last 3 turns. So, $r_3 = \hat{p}_{CCC},
r_2 = \hat{p}_{CCD} =  \hat{p}_{CDC} = \hat{p}_{DCC}, r_1 = \hat{p}_{CDD} =
\hat{p}_{DCD} =  \hat{p}_{DDC}, r_0 = \hat{p}_{CCC}$ and $\mathbf{\hat{p}}=(r_0,
r_1, r_2, r_3)$. Then, conditions~(\ref{eq:three_bit_conditions}), the
conditions for being a partner strategy become:

\begin{equation}\label{eq:counting_three_bit_conditions}
  \displaystyle r_3 < 1, \quad r_2 < 1-\frac{c}{3b}, \quad r_1 < 1-\frac{2c}{3b} ~~and~~ r_0 < 1\!-\! \frac{c}{b}.
\end{equation}


\subsection{$n-$bit counting partner strategies}

In the case of counting reactive strategies, we observe a pattern in the
conditions they must satisfy to be partner strategies. We show that for an $n-$bit
counting reactive strategy to be a partner strategy, the strategy's entries must
satisfy the conditions:

\begin{align*}
    r_{n}   \leq & 1 \\
    r_{n-1} \leq & 1  - \frac{(n - 1)}{n} \times \frac{c}{b}\\
    r_{n-2} \leq & 1  - \frac{(n - 2)}{n} \times \frac{c}{b}\\
    & \vdots \\
    r_{0} \leq &  1  - \frac{c}{b}\\
\end{align*}

\subsection{Evolutionary Dynamics}

\section{Discussion}

~\\
\bibliography{bibliography.bib}

\end{document}