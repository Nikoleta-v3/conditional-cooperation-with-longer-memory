\documentclass{article}

\usepackage{minitoc}
\usepackage{tabularx}
\usepackage{booktabs}
\usepackage{amssymb}
\usepackage{graphics}
\usepackage{hyperref}
\usepackage{xcolor}
\usepackage{amsmath}
\usepackage{blkarray}

\usepackage[margin=2.5cm, includefoot, footskip=30pt]{geometry}
\pagestyle{plain}
\setlength{\parindent}{0em}
\setlength{\parskip}{1em}
\renewcommand{\baselinestretch}{1}

\newcommand{\nikoleta}[1]{\textcolor{orange}{{\bf NG:} #1}}

\title{Two-bits (and more) reactive strategies in repeated games}

\author{Nikoleta E. Glynatsi, Christian Hilbe, Martin Nowak}
\date{}

\begin{document}

\maketitle

We explore infinitely repeated (\(2 \times 2\)) symmetric games. Players use the
set reactive strategies to make decision at each turn. We consider the repeated
game where at each turn players, simultaneously and independently, decide to
cooperate or to defect. In the case of two players, the payoffs are the
following,

\begin{equation}\label{eq:payoffs}
    \begin{blockarray}{ccc}
        & \text{cooperate} & \text{defect} \\
        \begin{block}{c(cc)}
            \text{cooperate} & b - c & -c \\
            \text{defect} & b & 0 \\
        \end{block}
    \end{blockarray}
  \end{equation}


Strategies of the reactive set only take into account the actions of the
co-player. The most well studied reactive strategies are those that take into
account only the last turn of the opponent. Here will reefer to these as
\textit{one-bit reactive strategies}. A one-bit reactive strategy is written as
\(p = (p_{C}, p_{D})\) where \(p_{C}\) is the probability of cooperating after
the co-player has cooperated and \(p_{D}\) after they defected.

The play of reactive strategies can be modelled as a Markov chain. In the case
of the one-bit reactive strategies, there are only 4 possibles states \(CC, CD,
DC, DD\) and the transition matrix is given by,

\begin{equation}
\displaystyle \left[\begin{matrix}p_{1} q_{1} & p_{1} \left(1 - q_{1}\right) & q_{1} \left(1 - p_{1}\right) & \left(1 - p_{1}\right) \left(1 - q_{1}\right)\\p_{2} q_{1} & p_{2} \left(1 - q_{1}\right) & q_{1} \left(1 - p_{2}\right) & \left(1 - p_{2}\right) \left(1 - q_{1}\right)\\p_{1} q_{2} & p_{1} \left(1 - q_{2}\right) & q_{2} \left(1 - p_{1}\right) & \left(1 - p_{1}\right) \left(1 - q_{2}\right)\\p_{2} q_{2} & p_{2} \left(1 - q_{2}\right) & q_{2} \left(1 - p_{2}\right) & \left(1 - p_{2}\right) \left(1 - q_{2}\right)\end{matrix}\right]
\end{equation}


Here we explore of \textit{two-bit reactive strategies}. The transition matrix
is now given by,


Computing the stationary distribution of this matrix (analytically) is not
trivia.

\subsection{Analytical analysis}


\begin{align}
  y &= \begin{bmatrix}
         x_{1} \\
         x_{2} \\
         \vdots \\
         x_{m}
       \end{bmatrix}
\end{align}


\begin{align}
[   &  p_{1}^{2} q_{1}^{2} \\
    &  p_{1}^{2} q_{1} \left(1 -q_{1}\right) \\
    &  p_{1} q_{1}^{2} \left(1 - p_{1}\right) \\ 
    &  p_{1} q_{1} \left(p_{1} -1\right) \left(q_{1} - 1\right) \\
    &  p_{1}^{2} q_{1} \left(1 -q_{1}\right) \\
    &  p_{1}^{2} \left(q_{1} - 1\right)^{2}\\
    &  p_{1} q_{1} \left(p_{1} - 1\right) \left(q_{1} - 1\right)\\
    &  - p_{1} \left(p_{1} - 1\right) \left(q_{1} - 1\right)^{2}\\
    & p_{1} q_{1}^{2} \left(1 - p_{1}\right)\\
    & p_{1} q_{1} \left(p_{1} - 1\right) \left(q_{1} - 1\right)\\
    & q_{1}^{2} \left(p_{1} - 1\right)^{2} \\
    & - q_{1}\left(p_{1} - 1\right)^{2} \left(q_{1} - 1\right)\\
    & p_{1} q_{1} \left(p_{1} - 1\right) \left(q_{1} - 1\right)\\
    & - p_{1} \left(p_{1} - 1\right) \left(q_{1} - 1\right)^{2}\\
    & - q_{1} \left(p_{1} - 1\right)^{2} \left(q_{1} - 1\right)\\
    & \left(p_{1} - 1\right)^{2} \left(q_{1} - 1\right)^{2}]
\end{align}


\subsection{Numerical analysis}

For proof that our formulation is correct to the Jupyter Notebook ``Numerical simulations''.

\subsection{Evolutionary simulations}

An evolutionary approach based on Nowak and Imphof gives the following
results when we vary the benefit \(c\).


\end{document}